\chapter{A Small GNU Autotools Project}\label{C_A_Small_GNU_Autotools_Project}


This chapter introduces a small--but real--worked example, to illustrate 
some of the features, and highlight some of the pitfalls ({\MjQ\cH184}\z{\MtQ\cH201}; {\MeQ\cH11}\z{\MeQ\cH80}) , of the 
GNU Autotools discussed \underline{so far} ({\MaQ\cH131}\z{\MbQ\cH241}\z{\MaQ\cH135}\z{\MbQ\cH209}\z{\MbQ\cH162}). All of the source 
can be downloaded from 
the book's web page\footnote{http://sources.redhat.com/autobook/}.
The text is peppered ({\MaQ\cH85}\z{\MdQ\cH48}\z{\MbQ\cH197} [ (+with )]) with my own pet ({\MbQ\cH41}\z{\MbQ\cH60}\z{\MbQ\cH237}) ideas,
accumulated ({\MhQ\cH218}\z{\McQ\cH9}, {\McQ\cH9}\z{\MiQ\cH27}; {\McQ\cH9}\z{\MxQ\cH217}) over a several years of working with the 
GNU Autotools and you should be able to easily apply these to your 
own projects. I will begin by describing some of the choices and problems I 
encountered during the early stages of the development of this project. 
Then by way of illustration of the issues covered, move on to showing 
you a general infrastructure ({\MeQ\cH33}\z{\MhQ\cH128}\z{\MbQ\cH29}\z{\McQ\cH113}) that I use as the basis for all 
of my own projects, followed by the specifics ({\MiQ\cH204}\z{\MbQ\cH56}, {\McQ\cH30}\z{\McQ\cH20}) of the 
implementation of a portable command line shell library. This 
chapter then finishes with a sample shell application that uses that library. 


Later, in chapter \ref{C_A_Large_GNU_Autotools_Project}
A Large GNU Autotools Project, page \pageref{C_A_Large_GNU_Autotools_Project}
and chapter \ref{C_A_Complex_GNU_Autotools_Project} A Complex GNU 
Autotools Project,
page \pageref{C_A_Complex_GNU_Autotools_Project}, the example introduced 
here will be gradually ({\MjQ\cH77}\z{\MbQ\cH165}\z{\MaQ\cH203}, {\MgQ\cH218}\z{\MgQ\cH218}\z{\MaQ\cH203}) expanded as new features 
of GNU Autotools are revealed ({\MbQ\cH4}\z{\MbQ\cH219}, {\MjQ\cH233}\z{\MjQ\cH208}\z{\MaQ\cH124}) .

\section{GNU Autotools in Practice}


This section details some of the specific problems I encountered when 
starting this project, and is representative ({\MaQ\cH72}\z{\McQ\cH91}\z{\MbQ\cH52}\z{\MbQ\cH237}, {\MdQ\cH116}\z{\MeQ\cH25}\z{\MbQ\cH237}) of 
the sorts ({\McQ\cH7}\z{\McQ\cH233}, {\MaQ\cH185}\z{\McQ\cH7}, {\McQ\cH233}\z{\MeQ\cH25}, {\MbQ\cH52}\z{\McQ\cH148}; {\MbQ\cH52}\z{\MbQ\cH143}) of things you are likely ({\MbQ\cH39}\z{\MaQ\cH170}\z{\McQ\cH63}\z{\MbQ\cH237})
to want to do in projects of your own, but for which the correct solution 
may not be immediately evident ({\MbQ\cH112}\z{\MjQ\cH233}\z{\MbQ\cH237}; {\MbQ\cH112}\z{\MbQ\cH235}\z{\MbQ\cH237}). You can always refer back to 
this section for some inspiration ({\MjQ\cH209}\z{\MbQ\cH62}) if you come across similar situations.
I will talk about some of the decisions I made about the structure of the 
project, and also the trade-offs for the other side of the argument --
you might find the opposite choice to theone I make here is 
more relevant a particular project of yours. 

\subsection{Project Directory Structure}\label{pds}


Before starting to write code for any project, you need to decide on 
the directory structure you will use to organise (=organized) the code.
I like to build each component of a project in its own subdirectory,
and to keep the configuration sources separate from the source code.
The great majority ({\MaQ\cH214}\z{\MbQ\cH98}, {\McQ\cH172}\z{\MaQ\cH154}\z{\MbQ\cH98}, {\MaQ\cH215}\z{\MaQ\cH214}\z{\MbQ\cH98}) of GNU projects I 
have seen use a similar method, so adopting it yourself will likely 
make your project more familiar to your developers 
by association ({\MaQ\cH155}\z{\MbQ\cH125}, {\MaQ\cH116}\z{\MbQ\cH125}, {\MbQ\cH254}\z{\MaQ\cH200}). 


The top level directory is used for configuration files, such as `configure'
and `aclocal.m4', and for a few other sundry ({\MaQ\cH174}\z{\MbQ\cH31}\z{\MaQ\cH174}\z{\MbQ\cH154}\z{\MbQ\cH237}, {\MjQ\cH198}\z{\MbQ\cH237}, {\MaQ\cH174}\z{\McQ\cH7}\z{\MbQ\cH237})
files, `README' and a copy of the project license for example. 


Any significant libraries will have a subdirectory of their own, containing all of the sources and headers for that library along with a `Makefile.am' and anything else that is specific to just that library. Libraries that are part of a small like group, a set of pluggable application modules for example, are kept together in a single directory. 


The sources and headers for the project's main application will be stored in yet another subdirectory, traditionally named `src'. There are other conventional directories your developers might expect too: A `doc' directory for project documentation; and a `test' directory for the project self test suite. 


To keep the project top-level directory as uncluttered ({\MaQ\cH46}\z{\MdQ\cH126}\z{\MdQ\cH18}\z{\MbQ\cH237}; {\MbQ\cH99}\z{\MkQ\cH53}\z{\MbQ\cH237})
as possible, as I like to do, you can take advantage of
Autoconf's `AC\_{}CONFIG\_{}AUX\_{}DIR' by creating another durectory,
say `config', which will be used to store many of the GNU Autotools 
intermediate ({\MaQ\cH50}\z{\McQ\cH200}\z{\MbQ\cH237}, {\MbQ\cH2}\z{\MaQ\cH50}\z{\MbQ\cH237}; {\MaQ\cH50}\z{\MeQ\cH25}\z{\MbQ\cH237}) files, such as install-sh. I always
store all project specific Autoconf M4 macros to this same subdirectory. 


So, this is what you should start with:

\begin{Verbatim}[frame=single]
$ pwd
~/mypackage
$ ls -F
Makefile.am  config/     configure.in  lib/  test/
README       configure*  doc/          src/
\end{Verbatim}

\subsection{C Header Files}\label{chf}


There is a small amount of boiler-plate that should be added to all header files, not least of which is a small amount of code to prevent the contents of the header from being scanned multiple times. This is achieved by enclosing the entire file in a preprocessor conditional which evaluates to false after the first time it has been seen by the preprocessor. Traditionally, the macro used is in all upper case, and named after the installation path without the installation prefix. Imagine a header that will be intalled to `/usr/local/include/sys/foo.h', for example. The preprocessor code would be as follows: 


\begin{Verbatim}[frame=single]
#ifndef SYS_FOO_H
#define SYS_FOO_H 1
...
#endif /* !SYS_FOO_H */
\end{Verbatim}



\underline{Apart from} ({\McQ\cH206}\z{\McQ\cH199}; {\McQ\cH206}... {\MaQ\cH53}\z{\MaQ\cH213}) comments,
the entire content of the rest of this header file 
must be between these few lines. It is worth mentioning that inside the 
enclosing ifndef, the macro SYS\_{}FOO\_{}H must be defined before any other 
files are \#included. It is a common mistake to not define that macro until 
the end of the file, but mutual ({\MbQ\cH243}\z{\MdQ\cH19}\z{\MbQ\cH237}, {\MfQ\cH11}\z{\MbQ\cH164}\z{\MbQ\cH237}) dependency cycles are 
only stalled ({\MfQ\cH122}... {\McQ\cH201}\z{\MaQ\cH112}\z{\MhQ\cH69}\z{\MiQ\cH74}) if the 
guard macro is defined before the \#include which starts that
cycle\footnote{An \#include cycle is the situation where file `a.h' \#includes
file `b.h', and `b.h' \#includes file `a.h' -- either directly or through 
some longer chain of \#includes. }. 


If a header is designed to be installed, it must \#include other installed 
project headers from the local tree using angle-brackets. There are some 
implications to working like this: 

\begin{itemize}
\item You must be careful that the names of header file directories in the 
source tree match the names of the directories in the install tree.
For example, when I plan to install the aforementioned ({\MaQ\cH135}\z{\McQ\cH222}\z{\MbQ\cH84}\z{\MaQ\cH163}\z{\MbQ\cH237}, {\MaQ\cH44}\z{\MjQ\cH72}\z{\MbQ\cH237})
`foo.h' to `/usr/local/include/project/foo.h', from which it will be included 
using `\verb+#include <project/foo.h>+', then in order for the same 
include line to work in the source tree, I must name the source directory 
it is installed from `project' too, or other headers which use it will not 
be able to find it until after it has been installed. 

\item When you come to developing the next version of a project laid out 
in this way, you must be careful about finding the correct header. Automake 
\underline{takes care of} ({\MbQ\cH206}\z{\MjQ\cH232}, {\McQ\cH84}\z{\MbQ\cH220}) that for you by using `-I' options 
that force the compiler 
to look for uninstalled headers in the current source directory before 
searching the system directories for installed headers of the same name. 

\item You don't have to install all of your headers to `/usr/include' -- you 
can use subdirectories. And all without having to rewrite the headers at 
install time. 
\end{itemize}

\subsection{C++ Compilers}


In order for a C++ program to use a library compiled with a C compiler,
it is neccessary for any symbols exported from the C library to be declared
between \verb+`extern "C" {'+ and \verb+`}'+. This code is important,
because a C++ compiler mangles\footnote{For an explanation of name mangling 
See chapter \ref{Writing_Portable_CPP_with_GNU_Autotools} Writing 
Portable C++ with GNU Autotools,
page \pageref{Writing_Portable_CPP_with_GNU_Autotools}.}
all variable and function names, where as a C compiler does not. On the 
other hand, a C compiler will not understand these lines, so you must be 
careful to make them invisible to the C compiler. 


Sometimes you will see this method used, written out in long hand in every installed header file, like this: 

\begin{Verbatim}[frame=single]
#ifdef __cplusplus
extern "C" {
#endif

...

#ifdef __cplusplus
}
#endif
\end{Verbatim}

But that is a lot of unnecessary typing if you have a few dozen ({\MbQ\cH22}\z{\MdQ\cH1}, {\McQ\cH114}\z{\MaQ\cH214})
headers in your project. Also the additional braces tend to confuse text 
editors, such as emacs, which do automatic source indentation ({\McQ\cH237}\z{\McQ\cH87}\z{\MhQ\cH242}\z{\MfQ\cH167}{\MaQ\cH2} {\MhQ\cH242}\z{\MbQ\cH143})
based on brace characters. 


Far better, then, to declare them as macros in a common header file,
and use the macros in your headers: 



\begin{Verbatim}[frame=single]
#ifdef __cplusplus
#  define BEGIN_C_DECLS extern "C" {
#  define END_C_DECLS   }
#else /* !__cplusplus */
#  define BEGIN_C_DECLS
#  define END_C_DECLS
#endif /* __cplusplus */
\end{Verbatim}



I have seen several projects that name such macros with a leading 
underscore ({\MaQ\cH202}... {\MaQ\cH45}\z{\MdQ\cH149}\z{\McQ\cH40}) -- `\_{}BEGIN\_{}C\_{}DECLS'. Any symbol with a 
leading underscore 
is reserved for use by the compiler implementation, so you shouldn't name 
any symbols of your own in this way. By way of example, I recently 
ported the Small\footnote{http://www.compuphase.com/small.htm }
language compiler to Unix, and almost all of the 
work was writing a Perl script to rename huge numbers of symbols in the 
compiler's reserved namespace to something more
sensible ({\MbQ\cH112}\z{\MgQ\cH4}\z{\MbQ\cH237}; {\MaQ\cH175}\z{\MbQ\cH56}\z{\MbQ\cH220}\z{\MbQ\cH237}, {\MbQ\cH60}\z{\McQ\cH133}\z{\MaQ\cH131}\z{\MbQ\cH237}, {\MaQ\cH244}\z{\McQ\cH103}\z{\MaQ\cH131}\z{\MbQ\cH237}) so that GCC could 
even parse the sources. Small was originally developed on Windows, and the 
author had used a lot of symbols with a leading underscore. Although his 
symbol names didn't \underline{clash with} ({\McQ\cH68}... {\MiQ\cH167}\z{\McQ\cH12}) his own compiler,
in some cases they were 
the same as symbols used by GCC. 


\subsection{Function Definitions}


As a stylistic ({\MrQ\cH68}\z{\MhQ\cH127}\z{\MbQ\cH100}\z{\McQ\cH241}\z{\MbQ\cH237}; {\MbQ\cH100}\z{\McQ\cH241}\z{\McQ\cH127}\z{\MbQ\cH237}) convention, the return types for all 
function definitions should be on a separate line. The main reason for this is 
that it makes it very easy to find the functions in source file, by looking 
for a single identifier at the start of a line followed by an open 
parenthesis ({\MaQ\cH198}\z{\MfQ\cH142}\z{\McQ\cH85}) : 


\begin{Verbatim}[frame=single]
$ egrep '^[_a-zA-Z][_a-zA-Z0-9]*[ \t]*\(' error.c
set_program_name (const char *path)
error (int exit_status, const char *mode, const char *message)
sic_warning (const char *message)
sic_error (const char *message)
sic_fatal (const char *message)
\end{Verbatim}



There are emacs lisp functions and various code analysis tools, such as 
ansi2knr (see subsection \ref{ss_k_r_compilers} K\&R Compilers,
page \pageref{ss_k_r_compilers}), which \underline{rely on} ({\MaQ\cH89}\z{\MjQ\cH10}, {\MaQ\cH89}\z{\MjQ\cH212})
this formatting convention, too. Even if you don't use those tools yourself,
your fellow ({\MdQ\cH35}\z{\MdQ\cH45}; {\MaQ\cH176}\z{\MaQ\cH57}) developers might like to,
so it is a good convention to adopt. 

\subsection[Fallback Function Implementations]{Fallback ({\MlQ\cH168}\z{\McQ\cH162}, {\MaQ\cH170}\z{\MaQ\cH89}\z{\MjQ\cH212}\z{\MbQ\cH237}\z{\MbQ\cH137}\z{\McQ\cH97}) Function Implementations}\label{ffi}


Due to the huge number of Unix varieties in common use today, many of the C 
library functions that you \underline{take for} ({\McQ\cH119}\z{\MbQ\cH209}) granted ({\MfQ\cH164}\z{\MaQ\cH56}) on 
your prefered development platform are very likely missing from some of 
the architectures you would like your code to compile on.
Fundamentally ({\MeQ\cH33}\z{\MhQ\cH128}\z{\MaQ\cH203}; {\MgQ\cH59}\z{\MbQ\cH133}\z{\MaQ\cH203}; {\McQ\cH189}\z{\McQ\cH98}\z{\MaQ\cH203}) there are two ways to cope with this: 

\begin{itemize}
\item Use only the few library calls that are available everywhere. In 
reality this is not actually possible because there are two lowest common 
denominators ({\MaQ\cH125}\z{\MgQ\cH130}; {\MaQ\cH183}\z{\MaQ\cH177}\z{\McQ\cH54}; {\McQ\cH150}\z{\MgQ\cH196}) with mutually ({\MdQ\cH19}\z{\MbQ\cH243}, {\MfQ\cH11}\z{\MbQ\cH164})
exclusive ({\MfQ\cH167}\z{\MaQ\cH213}\z{\MbQ\cH237}; {\McQ\cH206}\z{\MaQ\cH213}\z{\MbQ\cH237}; {\MaQ\cH114}\z{\McQ\cH182}\z{\MbQ\cH237}; {\MdQ\cH241}\z{\McQ\cH248}\z{\MbQ\cH237}) APIs,
one \underline{rooted in} ({\MbQ\cH191}\z{\MgQ\cH84}\z{\MbQ\cH107}; {\MgQ\cH59}\z{\MgQ\cH196}\z{\MbQ\cH107}) BSD Unix
(`bcopy', `rindex') and the other in SYSV Unix (`memcpy', `strrchr'). The 
only way to deal with this is to define one API 
\underline{in terms of} ({\MaQ\cH255}... {\McQ\cH55}\z{\McQ\cH127}; {\MaQ\cH202}... {\MbQ\cH106}\z{\McQ\cH222}) the other 
using the preprocessor. The newer POSIX standard 
deprecates ({\McQ\cH59}\z{\MbQ\cH112}\z{\MaQ\cH46}\z{\MjQ\cH14}\z{\MbQ\cH65}; {\MaQ\cH165}\z{\MaQ\cH250}; {\MjQ\cH47}\z{\McQ\cH101}) many of the BSD 
originated calls (with exceptions such as the BSD socket API). Even on 
non-POSIX platforms, there has been so much cross pollination ({\MfQ\cH164}\z{\MhQ\cH195} ({\MaQ\cH84}\z{\MbQ\cH224}) )
that often both varieties of a given call may be provided, however you 
would be wise to write your code using POSIX endorsed ({\MjQ\cH14}\z{\MaQ\cH176}; {\McQ\cH119}\z{\MaQ\cH170}) calls,
and where they are missing, define them in terms of whatever the 
host platform provides. 

This approach requires a lot of knowledge about various system libraries and 
standards documents, and can leave you with reams ({\MaQ\cH215}\z{\McQ\cH190}) of preprocessor 
code to handle the differences between APIS. You will also need to perform a 
lot of checking in `configure.in' to figure out which calls are available.
For example, to allow the rest of your code to use the `strcpy' call with 
impunity ( ({\MfQ\cH96}\z{\MiQ\cH3}{\MaQ\cH2} {\MfQ\cH184}\z{\MaQ\cH220}{\MaQ\cH2} {\MdQ\cH97}\z{\MaQ\cH239}\z{\McQ\cH16}\z{\MbQ\cH237}) {\MaQ\cH110}\z{\McQ\cH206}) , you would need the 
following code in `configure.in': 

\begin{Verbatim}[frame=single]
AC_CHECK_FUNCS(strcpy bcopy)
\end{Verbatim}

And the following preprocessor code in a header file that is seen by 
every source file: 

\begin{Verbatim}[frame=single]
#if !HAVE_STRCPY
#  if HAVE_BCOPY
#    define strcpy(dest, src) \
      bcopy (src, dest, 1 + strlen (src))
#  else /* !HAVE_BCOPY */
     error no strcpy or bcopy
#  endif /* HAVE_BCOPY */
#endif /* HAVE_STRCPY */
\end{Verbatim}

\item Alternatively you could provide your own fallback implementations of 
function calls you know are missing on some platforms. In practice you don't 
need to be as knowledgable about problematic ({\MaQ\cH189}\z{\McQ\cH230}\z{\MbQ\cH237}; {\MhQ\cH79}\z{\McQ\cH216}\z{\MbQ\cH237}; {\MaQ\cH46}\z{\MbQ\cH252}\z{\MaQ\cH236}\z{\MbQ\cH237})
functions when using this approach.
You can look in GNU libiberty\footnote{Available at
ftp://sourceware.cygnus.com/pub/binutils/.} or Fran\c{c}ois Pinard's libit 
project\footnote{Distributed from http://www.iro.umontreal.ca/~pinard/libit.}
to see for which functions other GNU developers have needed to implement 
fallback code. The libit project is especially useful in this respect
({\McQ\cH201}\z{\MaQ\cH91}; {\MbQ\cH106}\z{\McQ\cH222}, {\McQ\cH79}\z{\MhQ\cH105}\z{\McQ\cH245}) as it comprises canonical ({\MbQ\cH153}\z{\MbQ\cH196}\z{\MbQ\cH237}; {\MbQ\cH158}\z{\MeQ\cH110}\z{\MbQ\cH237}) versions of 
fallback functions, and suitable Autoconf macros assembled from across the 
entire GNU project. I won't give an example of setting up your package to 
use this approach, since that is how I have chosen to structure the 
project described in this chapter. 
\end{itemize}

Rather than writing code to the lowest common denominator of system libraries,
I am a strong advocate ({\MbQ\cH84}\z{\MdQ\cH76}\z{\McQ\cH54}; {\MfQ\cH203}\z{\McQ\cH136}\z{\McQ\cH54}) of the latter school of thought in the 
majority ({\MaQ\cH214}\z{\MbQ\cH98}, {\McQ\cH172}\z{\MaQ\cH154}\z{\MbQ\cH98}, {\MaQ\cH215}\z{\MaQ\cH214}\z{\MbQ\cH98}) of cases. As with all things it pays to 
take a pragmatic ({\MaQ\cH142}\z{\MaQ\cH245}\z{\MbQ\cH237}; {\MaQ\cH245}\z{\MeQ\cH225}\z{\MbQ\cH237}) approach; don't be afraid of the 
\underline{middle ground} ({\MeQ\cH93}\z{\MaQ\cH155}, {\MaQ\cH50}\z{\McQ\cH200}\z{\McQ\cH13}\z{\MaQ\cH210}) -- weigh ({\McQ\cH53}\z{\MbQ\cH63}; {\MbQ\cH158}\z{\MiQ\cH169}) the options 
on a case by case basis.

\subsection{K\&R Compilers}\label{ss_k_r_compilers}

K\&R C is the name now used to describe the original C language specified 
by Brian Kernighan and Dennis Ritchie (hence, `K\&R'). I have yet to 
see a C compiler that doesn't support code written in the K\&R style, yet 
it has fallen very much into disuse ({\MaQ\cH46}\z{\MbQ\cH224}; {\MeQ\cH242}\z{\MgQ\cH74}) \underline{in favor of}
({\MjQ\cH14}\z{\MbQ\cH65}...; {\MbQ\cH88}\z{\MbQ\cH77}...; {\MbQ\cH127}\z{\MaQ\cH130}\z{\MbQ\cH107}...) the newer ANSI C standard.
Although it is increasingly common for vendors to unbundle their ANSI C 
compiler, the GCC project\footnote{GCC must be compilable by K\&R compilers so 
that it can be built and installed in an ANSI compiler free environment.}
is available for all of the architectures I have ever used. 


There are four differences between the two C standards: 

\begin{enumerate}
\item ANSI C expects full type specification in function prototypes, such as 
you might supply in a library header file:

\begin{Verbatim}[frame=single]
extern int functionname (const char *parameter1,
                         size_t parameter 2);
\end{Verbatim}

The nearest equivalent in K\&R style C is a forward declaration, which 
allows you to use a function before its corresponding definition: 

\begin{Verbatim}[frame=single]
extern int functionname ();
\end{Verbatim}

As you can imagine, K\&R has very bad type safety, and does not perform any checks that only function arguments of the correct type are used. 


\item The function headers of each function definition are written differently. Where you might see the following written in ANSI C: 

\begin{Verbatim}[frame=single]
int
functionname (const char *parameter1, size_t parameter2)
{
  ...
}
\end{Verbatim}



K\&R expects the parameter type declarations separately, like this: 

\begin{Verbatim}[frame=single]
int
functionname (parameter1, parameter2)
     const char *parameter1;
     size_t parameter2;
{
  ...
}
\end{Verbatim}

\item There is no concept of an untyped pointer in K\&R C. Where you might be used to seeing `void *' pointers in ANSI code, you are forced to overload the meaning of `char *' for K\&R compilers. 

\item Variadic functions are handled with a different API in K\&R C,
imported with `\verb+#include <varargs.h>+'. A K\&R variadic function definition looks like this:

\begin{Verbatim}[frame=single]
int
functionname (va_alist)
     va_dcl
{
  va_list ap;
  char *arg;

  va_start (ap);
  ...
  arg = va_arg (ap, char *);
  ...
  va_end (ap);

  return arg ? strlen (arg) : 0;
}
\end{Verbatim}

ANSI C provides a similar API, imported with `\verb+#include <stdarg.h>+',
though it cannot express a variadic function with no named arguments such 
as the one above. In practice, this isn't a problem since you always 
need at least one parameter, either to specify the total number of 
arguments somehow, or else to mark the end of the argument list. An 
ANSI variadic function definition looks like this: 

\begin{Verbatim}[frame=single]
int
functionname (char *format, ...)
{
  va_list ap;
  char *arg;

  va_start (ap, format);
  ...
  arg = va_arg (ap, char *);
  ...
  va_end (ap);

  return format ? strlen (format) : 0;
}

\end{Verbatim}

\end{enumerate}

Except in very rare cases where you are writing a low level project
(GCC for example), you probably don't need to worry about K\&R compilers 
too much. However, supporting them can be very easy, and if you are 
so inclined ({\MdQ\cH98}\z{\MaQ\cH178}\z{\MbQ\cH237}, {\MdQ\cH98}\z{\MfQ\cH233}\z{\MbQ\cH237}), can be handled either by employing the ansi2knr 
program supplied with Automake, or by careful use of the preprocessor. 


Using ansi2knr in your project is described in some detail in 
section `Automatic de-ANSI-fication' in The Automake Manual,
but \underline{boils down} ({\MmQ\cH107}\z{\MgQ\cH235}, {\MmQ\cH115}\z{\MgQ\cH235}, {\MeQ\cH57}\z{\MhQ\cH242}, {\McQ\cH21}\z{\MaQ\cH147}; {\MbQ\cH166}\z{\McQ\cH33}) to the following: 

\begin{itemize}
\item Add this macro to your `configure.in' file: 

\begin{Verbatim}[frame=single]
AM_C_PROTOTYPES
\end{Verbatim}

\item Rewrite the contents of `LIBOBJS' and/or `LTLIBOBJS' in the 
following fashion: 


\begin{Verbatim}[frame=single]
# This is necessary so that .o files in LIBOBJS are also
# built via the ANSI2KNR-filtering rules.
Xsed='sed -e "s/^X//"'
LIBOBJS=`echo X"$LIBOBJS"|\
  [$Xsed -e 's/\.[^.]* /.\$U& /g;s/\.[^.]*$/.\$U&/']`
\end{Verbatim}

\end{itemize}

Personally, I dislike this method, since every source file is filtered and 
rewritten with ANSI function prototypes and declarations converted to K\&R 
style adding a fair overhead in additional files in your build tree,
and in compilation time. This would be reasonable ({\MaQ\cH175}\z{\MbQ\cH220}\z{\MbQ\cH237}, {\MbQ\cH163}\z{\MbQ\cH231}\z{\MbQ\cH237}; {\MjQ\cH95}\z{\MbQ\cH231}\z{\MbQ\cH237})
were the abstraction sufficient to allow you to forget about K\&R entirely,
but ansi2knr is a simple program, and does not address any of the other
differences between compilers that I raised above, and it cannot handle macros 
in your function prototypes of definitions. If you decide to use ansi2knr in 
your project, you must make the decision before you write any code, and be 
aware of its limitations as you develop. 


For my own projects, I prefer to use a set of preprocessor macros 
\underline{along with} ({\McQ\cH68}... {\MaQ\cH202}\z{\McQ\cH248}\z{\McQ\cH150}; {\MaQ\cH202}... {\MaQ\cH74}\z{\MaQ\cH213}) a few stylistic
({\MrQ\cH68}\z{\MhQ\cH127}\z{\MbQ\cH100}\z{\McQ\cH241}\z{\MbQ\cH237}; {\MbQ\cH100}\z{\McQ\cH241}\z{\McQ\cH127}\z{\MbQ\cH237}) conventions so that all of the differences 
between K\&R and ANSI compilers are actually addressed, and so that the 
unfortunate few who have no access to an ANSI compiler (and who cannot 
use GCC for some reason) needn't suffer ({\MjQ\cH96}\z{\MaQ\cH167}; {\McQ\cH37}\z{\MgQ\cH118}) the overheads of ansi2knr. 


The four differences in style listed at the beginning of this subsection 
are addressed as follows: 

\begin{enumerate}
\item The function protoype argument lists are declared inside a PARAMS macro 
invocation so that K\&R compilers will still be able to compile the 
source tree. PARAMS removes ANSI argument lists from function prototypes for 
K\&R compilers. Some developers continue to use \_{}\_{}P for this purpose,
but strictly speaking, macros starting with `\_{}' (and especially `\_{}\_{}') 
are reserved for the compiler and the system headers, so using `PARAMS',
as follows, is safer: 

\begin{Verbatim}[frame=single]
#if __STDC__
#  ifndef NOPROTOS
#    define PARAMS(args)      args
#  endif
#endif
#ifndef PARAMS
#  define PARAMS(args)        ()
#endif
\end{Verbatim}

This macro is then used for all function declarations like this: 

\begin{Verbatim}[frame=single]
extern int functionname PARAMS((const char *parameter));
\end{Verbatim}

\item With the PARAMS macro is used for all function declarations, ANSI 
compilers are given all the type information they require to do full compile 
time type checking. The function definitions proper must then be declared 
in K\&R style so that K\&R compilers don't choke ({\MfQ\cH123}\z{\MaQ\cH132}, {\MeQ\cH57}\z{\MaQ\cH82}) on ANSI syntax.
There is a small amount of overhead in writing code this way, however: The ANSI compile time type checking can only work in conjunction ({\McQ\cH33}\z{\MaQ\cH175}; {\McQ\cH201}\z{\McQ\cH58}; {\McQ\cH169}\z{\MbQ\cH81})
with K\&R function definitions if it first sees an ANSI function prototype.
This forces you to develop the good habit of prototyping every single 
function in your project. Even the static ones.

\item The easiest way to work around the lack of void * pointers, is to define a new type that is conditionally set to void * for ANSI compilers, or char * for K\&R compilers. You should add the following to a common header file: 

\begin{Verbatim}[frame=single]
#if __STDC__
typedef void *void_ptr;
#else /* !__STDC__ */
typedef char *void_ptr;
#endif /* __STDC__ */
\end{Verbatim}

\item The difference between the two variadic
function\marginpar{variadic function
{\MaQ\cH223}\z{\MdQ\cH100}\z{\MbQ\cH117}\z{\MaQ\cH161}\z{\MbQ\cH98}\z{\MaQ\cH46}\z{\MeQ\cH9}\z{\MaQ\cH236}\z{\MbQ\cH237}\z{\MdQ\cH131}\z{\MbQ\cH31}} APIs pose ({\McQ\cH168}\z{\MbQ\cH65}, {\MeQ\cH249}\z{\McQ\cH150})
a stickier ({\MlQ\cH234}\z{\MbQ\cH71}\z{\MbQ\cH237}, {\MkQ\cH44}\z{\MhQ\cH4}\z{\MbQ\cH237}) problem, and the solution is ugly. But it does work.
First you must check for the headers in `configure.in': 

\begin{Verbatim}[frame=single]
AC_CHECK_HEADERS(stdarg.h varargs.h, break)
\end{Verbatim}

Having done this, add the following code to a common header file: 

\begin{Verbatim}[frame=single]
#if HAVE_STDARG_H
#  include <stdarg.h>
#  define VA_START(a, f)        va_start(a, f)
#else
#  if HAVE_VARARGS_H
#    include <varargs.h>
#    define VA_START(a, f)      va_start(a)
#  endif
#endif
#ifndef VA_START
  error no variadic api
#endif
\end{Verbatim}

You must now supply each variadic function with both a K\&R and an ANSI 
definition, like this: 

\begin{Verbatim}[frame=single]
int
#if HAVE_STDARG_H
functionname (const char *format, ...)
#else
functionname (format, va_alist)
     const char *format;
     va_dcl
#endif
{
  va_alist ap;
  char *arg;

  VA_START (ap, format);
  ...
  arg = va_arg (ap, char *);
  ...
  va_end (ap);

  return arg : strlen (arg) ? 0;
}
\end{Verbatim}

\end{enumerate}

\section{A Simple Shell Builders Library}

An application which most developers \underline{try their hand} ({\MdQ\cH254}\z{\McQ\cH116}) at
\underline{sooner or later} ({\MjQ\cH98}\z{\MbQ\cH111}, {\MbQ\cH111}\z{\MbQ\cH119}) is 
a Unix shell. There is a lot of functionality common to all traditional 
command line shells, which I thought I would push into a portable 
library to get you over the first hurdle ({\MjQ\cH190}\z{\MmQ\cH243}, {\MeQ\cH8}\z{\McQ\cH216}) when that 
moment is upon you. Before elabourating ({\MmQ\cH109}\z{\McQ\cH145}\z{\McQ\cH75}\z{\MbQ\cH45}\z{\MbQ\cH237}, {\MjQ\cH58}\z{\MdQ\cH163}\z{\MbQ\cH237}) on any of this I 
need to name the project. I've called it \textit{sic}, from the Latin 
\textit{so it is}, because like all good project names it is somewhat 
pretentious ({\McQ\cH65}\z{\McQ\cH138}\z{\MbQ\cH237}; {\McQ\cH65}\z{\MaQ\cH183}\z{\MaQ\cH46}\z{\MaQ\cH123}\z{\MbQ\cH237}; {\MhQ\cH35}\z{\MeQ\cH89}\z{\MbQ\cH237}) and it \underline{lends itself to}
({\MjQ\cH95}\z{\MaQ\cH175}) the recursive acronym ({\McQ\cH237}\z{\MaQ\cH229}\z{\MgQ\cH130}\z{\MhQ\cH242}\z{\MhQ\cH73}\z{\MaQ\cH229}) sic is cumulative ({\MgQ\cH218}\z{\MaQ\cH211}\z{\MbQ\cH237}; {\MiQ\cH122}\z{\McQ\cH9}\z{\MbQ\cH237},
{\MhQ\cH218}\z{\McQ\cH108}\z{\MbQ\cH237}; {\MhQ\cH218}\z{\McQ\cH108}\z{\McQ\cH27}\z{\MaQ\cH130}\z{\MbQ\cH237}). 

The gory ({\MiQ\cH166}\z{\MgQ\cH139}\z{\MbQ\cH237}, {\MaQ\cH73}\z{\MaQ\cH65}\z{\MgQ\cH132}\z{\MkQ\cH12}\z{\MlQ\cH92}\z{\MbQ\cH205}\z{\MbQ\cH237}, {\MgQ\cH123}\z{\MoQ\cH117}\z{\MbQ\cH237}) detail of the minutae of the source 
is beyond the scope of this book, but to convey ({\McQ\cH171}\z{\McQ\cH163}, {\MfQ\cH186}\z{\McQ\cH171}, {\McQ\cH158}\z{\McQ\cH171}) a feel 
for the need for Sic, some of the goals which influenced ({\MfQ\cH9}\z{\McQ\cH224}, {\MaQ\cH84}\z{\MbQ\cH224})
the design follow: 

\begin{itemize}
\item Sic must be very small so that, in addition to being used as the 
basis for a \underline{full blown} ({\MaQ\cH234}\z{\MaQ\cH114}\z{\MbQ\cH65}\z{\MhQ\cH7}\z{\MbQ\cH237}; {\MaQ\cH107}\z{\MaQ\cH125}\z{\MbQ\cH234}\z{\MbQ\cH4}\z{\MbQ\cH237}; {\MaQ\cH120}\z{\MbQ\cH127}\z{\McQ\cH248}\z{\MaQ\cH126}\z{\MbQ\cH212}\z{\MfQ\cH23}\z{\MbQ\cH237})
shell, it can be linked (unadorned ({\MbQ\cH132}\z{\McQ\cH37}\z{\McQ\cH95}\z{\MjQ\cH242}\z{\MbQ\cH237}; {\MaQ\cH159}\z{\MaQ\cH86}\z{\MbQ\cH237}; {\MgQ\cH98}\z{\MhQ\cH214}\z{\MbQ\cH237}) ) into an 
application and used for trivial ({\MmQ\cH166}\z{\McQ\cH30}\z{\MbQ\cH237}; {\MaQ\cH46}\z{\McQ\cH189}\z{\McQ\cH98}\z{\MbQ\cH237}; {\MbQ\cH204}\z{\MaQ\cH103}\z{\MdQ\cH77}\z{\MbQ\cH237}) tasks,
such as reading startup configuration. 

\item It must not be tied to a particular syntax or set of reserved words.
If you use it to read your startup configuration, I don't want to force 
you to use my syntax and commands. 

\item The boundary ({\MjQ\cH105}\z{\MbQ\cH228}, {\MaQ\cH125}\z{\MbQ\cH228}\z{\McQ\cH40}) between the library (`libsic') and the 
application must be well defined. Sic will take strings of characters as input,
and internally parse and evaluate them according to registered commands and 
syntax, returning results or diagnostics ( ({\MbQ\cH224}\z{\MaQ\cH84}\z{\MaQ\cH192}\z{\MbQ\cH98}) {\MiQ\cH196}\z{\MbQ\cH105}\z{\MbQ\cH182}) as appropriate. 

\item It must be extremely portable -- that is what I am trying to 
illustrate here, after all. 
\end{itemize}

\subsection[Portability Infrastructure]{Portability Infrastructure
({\MaQ\cH116}\z{\MaQ\cH117}\z{\MbQ\cH29}\z{\McQ\cH113} ({\MaQ\cH224}\z{\McQ\cH196}\z{\McQ\cH152}{\MaQ\cH2} {\MaQ\cH116}\z{\McQ\cH152}{\MaQ\cH2} {\MaQ\cH45}\z{\MbQ\cH174}\z{\McQ\cH173}\z{\McQ\cH16}), {\MeQ\cH33}\z{\MhQ\cH128}\z{\MbQ\cH29}\z{\McQ\cH113})}

As I explained in \ref{pds} (page \pageref{pds}) Project Directory Structure,
I'll first create the 
project directories, a toplevel dirctory and a subdirectory to put the library 
sources into. I want to install the library header files 
to `/usr/local/include/sic', so the library subdirectory must be named 
appropriately. See subsubsection \ref{chf} (page \pageref{chf}) C Header Files. 


\begin{Verbatim}[frame=single]
$ mkdir sic
$ mkdir sic/sic
$ cd sic/sic
\end{Verbatim}

I will describe the files I add in this section in more detail than the 
project specific sources, because they comprise ({\MdQ\cH168}\z{\MdQ\cH214}, {\MdQ\cH168}\z{\MfQ\cH142}) an infrastructure 
that I use relatively unchanged for all of my GNU Autotools projects. You 
could keep an archive of these files, and use them as a starting point each 
time you begin a new project of your own. 

\subsubsection{Error Management}

A good place to start with any project design is the error management facility. In Sic I will use a simple group of functions to display simple error messages. Here is `sic/error.h': 

\begin{Verbatim}[frame=single]
#ifndef SIC_ERROR_H
#define SIC_ERROR_H 1

#include <sic/common.h>

BEGIN_C_DECLS

extern const char *program_name;
extern void set_program_name (const char *argv0);

extern void sic_warning      (const char *message);
extern void sic_error        (const char *message);
extern void sic_fatal        (const char *message);

END_C_DECLS

#endif /* !SIC_ERROR_H */
\end{Verbatim}

This header file follows the principles \underline{set out} ({\McQ\cH199}\z{\MaQ\cH225})
in \ref{chf} C Header Files (page \pageref{chf}).


I am storing the program\_{}name variable in the library that uses it, so that 
I can be sure that the library will build on architectures that don't 
allow undefined symbols in libraries\footnote{AIX and Windows being the main 
culprits.}. 


Keeping those preprocessor macro definitions designed to aid code portability 
together (in a single file), is a good way to maintain the readability of 
the rest of the code. For this project I will put that code in `common.h': 

\begin{Verbatim}[frame=single]
#ifndef SIC_COMMON_H
#define SIC_COMMON_H 1

#if HAVE_CONFIG_H
#  include <sic/config.h>
#endif

#include <stdio.h>
#include <sys/types.h>

#if STDC_HEADERS
#  include <stdlib.h>
#  include <string.h>
#elif HAVE_STRINGS_H
#  include <strings.h>
#endif /*STDC_HEADERS*/

#if HAVE_UNISTD_H
#  include <unistd.h>
#endif

#if HAVE_ERRNO_H
#  include <errno.h>
#endif /*HAVE_ERRNO_H*/
#ifndef errno
/* Some systems #define this! */
extern int errno;
#endif

#endif /* !SIC_COMMON_H */
\end{Verbatim}

You may recognise ({\McQ\cH119}\z{\MaQ\cH124}, {\McQ\cH133}\z{\MaQ\cH129}; {\McQ\cH119}\z{\McQ\cH133}) some snippets ({\MdQ\cH145}\z{\MfQ\cH165}\z{\MhQ\cH121}\z{\MbQ\cH210}; {\MbQ\cH105}\z{\MbQ\cH210}) of code 
from the Autoconf manual here --- in particular the inclusion of the project 
`config.h', which will be generated shortly. Notice that I have been careful 
to conditionally include any headers which are not guaranteed ({\MbQ\cH46}\z{\MaQ\cH236}\z{\MbQ\cH237}, {\MiQ\cH39}\z{\MaQ\cH236}\z{\MbQ\cH237}) 
to exist on every architecture. The \underline{rule of thumb}
({\MaQ\cH245}\z{\MbQ\cH224}\z{\MbQ\cH106}\z{\MbQ\cH182}; {\MbQ\cH150}\z{\MgQ\cH191}\z{\MbQ\cH182}; {\McQ\cH26}\z{\MhQ\cH73}\z{\MbQ\cH237}\z{\MiQ\cH169}\z{\McQ\cH190}, {\MeQ\cH33}\z{\MbQ\cH133}\z{\MaQ\cH159}\z{\MaQ\cH134}; {\McQ\cH37}\z{\McQ\cH240}\z{\MbQ\cH182}\z{\MaQ\cH134}; {\McQ\cH87}\z{\MaQ\cH57}\z{\MbQ\cH182}\z{\MaQ\cH134}) here is that 
only `stdio.h' is ubiquitous ({\MaQ\cH131}\z{\McQ\cH84}\z{\MaQ\cH230}\z{\MaQ\cH202}\z{\MbQ\cH237}, {\MgQ\cH0}\z{\MjQ\cH90}\z{\MaQ\cH230}\z{\MaQ\cH202}\z{\MbQ\cH237}) (though I have never 
heard of a machine that has no `sys/types.h'). You can find more details of 
some of these in section `Existing Tests' in \textit{The GNU Autoconf Manual}. 


Here is a little more code from `common.h': 

\begin{Verbatim}[frame=single]
#ifndef EXIT_SUCCESS
#  define EXIT_SUCCESS  0
#  define EXIT_FAILURE  1
#endif
\end{Verbatim}

The implementation of the error handling functions goes in `error.c' and is 
very straightforward ({\MbQ\cH112}\z{\MbQ\cH252}\z{\MbQ\cH237}, {\MiQ\cH39}\z{\MaQ\cH236}\z{\MbQ\cH237}) : 

\begin{Verbatim}[frame=single]
#if HAVE_CONFIG_H
#  include <sic/config.h>
#endif

#include "common.h"
#include "error.h"

static void error (int exit_status, const char *mode, 
                   const char *message);

static void
error (int exit_status, const char *mode, const char *message)
{
 fprintf (stderr,"%s: %s: %s.\n",program_name,mode, message);

 if (exit_status >= 0)
  exit (exit_status);
}

void sic_warning (const char *message)
{
 error (-1, "warning", message);
}

void sic_error (const char *message)
{
 error (-1, "ERROR", message);
}

void sic_fatal (const char *message)
{
 error (EXIT_FAILURE, "FATAL", message);
}
\end{Verbatim}

I also need a definition of program\_{}name; set\_{}program\_{}name 
copies the filename component of path into the exported data, program\_{}name.
The xstrdup function just calls strdup, but aborts if there is not enough 
memory to make the copy: 

\begin{Verbatim}[frame=single]
const char *program_name = NULL;

void set_program_name (const char *path)
{
 if (!program_name)
  program_name = xstrdup (basename (path));
}
\end{Verbatim}

\subsubsection{Memory Management}\label{mm}

A useful idiom common to many GNU projects is to wrap the memory management 
functions to localise \textit{out of memory handling}, naming them with 
an `x' prefix. By doing this, the rest of the project is relieved ({\MbQ\cH97}\z{\MbQ\cH200}; {\MbQ\cH97}\z{\MfQ\cH182};
{\McQ\cH106}\z{\MeQ\cH12}) of having to remember to check for `NULL' returns from the various 
memory functions. These wrappers use the error API to report memory 
exhaustion ({\MiQ\cH23}\z{\MhQ\cH96}; {\MgQ\cH44}\z{\MhQ\cH170}; {\McQ\cH23}\z{\MhQ\cH81}\z{\MaQ\cH137}\z{\MhQ\cH170}) and abort the program. I have placed the 
implementation code in `xmalloc.c': 

\begin{Verbatim}[frame=single]
#if HAVE_CONFIG_H
#  include <sic/config.h>
#endif

#include "common.h"
#include "error.h"

void * xmalloc (size_t num)
{
 void *new = malloc (num);
 if (!new)
  sic_fatal ("Memory exhausted");
 return new;
}

void * xrealloc (void *p, size_t num)
{
 void *new;

 if (!p)
  return xmalloc (num);

 new = realloc (p, num);
 if (!new)
  sic_fatal ("Memory exhausted");

 return new;
}

void * xcalloc (size_t num, size_t size)
{
 void *new = xmalloc (num * size);
 bzero (new, num * size);
 return new;
}
\end{Verbatim}

Notice in the code above, that xcalloc is implemented in terms of xmalloc,
since calloc itself is not available in some older C libraries. Also, the 
bzero function is actually deprecated ({\McQ\cH59}\z{\MbQ\cH112}\z{\MaQ\cH46}\z{\MjQ\cH14}\z{\MbQ\cH65}; {\MaQ\cH165}\z{\MaQ\cH250}; {\MjQ\cH47}\z{\McQ\cH101}) 
\underline{in favour of} ({\MjQ\cH14}\z{\MbQ\cH65}...; {\MbQ\cH88}\z{\MbQ\cH77}...; {\MbQ\cH127}\z{\MaQ\cH130}\z{\MbQ\cH107}...) memset in modern C 
libraries -- I'll explain how to take this into account later in 
\ref{boac} Beginnings of a `configure.in' (page \pageref{boac}). 


Rather than create a separate `xmalloc.h' file, which would need to 
be \#included from almost everywhere else, the logical place to declare 
these functions is in `common.h', since the wrappers will be called 
from most everywhere else in the code: 

\begin{Verbatim}[frame=single]
#ifdef __cplusplus
#  define BEGIN_C_DECLS         extern "C" {
#  define END_C_DECLS           }
#else
#  define BEGIN_C_DECLS
#  define END_C_DECLS
#endif

#define XCALLOC(type, num)                                  \
        ((type *) xcalloc ((num), sizeof(type)))
#define XMALLOC(type, num)                                  \
        ((type *) xmalloc ((num) * sizeof(type)))
#define XREALLOC(type, p, num)                              \
        ((type *) xrealloc ((p), (num) * sizeof(type)))
#define XFREE(stale)                      do {              \
              if (stale) { free (stale);  stale = 0; }      \
                                          } while (0)

BEGIN_C_DECLS

extern void *xcalloc    (size_t num, size_t size);
extern void *xmalloc    (size_t num);
extern void *xrealloc   (void *p, size_t num);
extern char *xstrdup    (const char *string);
extern char *xstrerror  (int errnum);

END_C_DECLS
\end{Verbatim}
By using the macros defined here, allocating and freeing heap memory is reduced from: 
\begin{Verbatim}[frame=single]
char **argv = (char **) xmalloc (sizeof (char *) * 3);
do_stuff (argv);
if (argv)
  free (argv);
\end{Verbatim}

to the simpler and more readable: 
\begin{Verbatim}[frame=single]
char **argv = XMALLOC (char *, 3);
do_stuff (argv);
XFREE (argv);
\end{Verbatim}
In the same spirit ({\McQ\cH23}\z{\MbQ\cH255}, {\MbQ\cH45}\z{\MjQ\cH209}) , I have borrowed ({\MaQ\cH97}\z{\MbQ\cH224}; {\MbQ\cH83}\z{\MbQ\cH224}) `xstrdup.c' 
and `xstrerror.c' from project GNU's libiberty. See subusbsection \ref{ffi}
Fallback Function Implementations (page \pageref{ffi}). 

\subsubsection{Generalised List Data Type}

In many C programs you will see various implementations and re-implementations 
of lists and stacks, each tied to its own particular project. It is 
surprisingly simple to write a catch-all implementation, as I have done here 
with a generalised list operation API in `list.h': 

\begin{Verbatim}[frame=single]
#ifndef SIC_LIST_H
#define SIC_LIST_H 1

#include <sic/common.h>

BEGIN_C_DECLS

typedef struct list 
{
 struct list *next;    /* chain forward pointer*/
 void *userdata;       /* incase you want to use raw Lists */
} List;

extern List *list_new       (void *userdata);
extern List *list_cons      (List *head, List *tail);
extern List *list_tail      (List *head);
extern size_t list_length   (List *head);

END_C_DECLS

#endif /* !SIC_LIST_H */
\end{Verbatim}

The trick is to ensure that any structures you want to chain together have 
their forward pointer in the first field. Having done that, the generic 
functions declared above can be used to manipulate any such chain by casting 
it to List * and back again as necessary. 

For example: 

\begin{Verbatim}
 struct foo 
 {
  struct foo *next;

  char *bar;
  struct baz *qux;
  ...
 };

 ...
  struct foo *foo_list = NULL;

  foo_list = (struct foo *) list_cons ((List *) new_foo (),
                                       (List *) foo_list);
 ...
\end{Verbatim}

The implementation of the list manipulation functions is in `list.c': 

\begin{Verbatim}[frame=single]
#include "list.h"

List * list_new (void *userdata)
{
 List *new = XMALLOC (List, 1);

 new->next = NULL;
 new->userdata = userdata;

 return new;
}

List * list_cons (List *head, List *tail)
{
 head->next = tail;
 return head;
}

List * list_tail (List *head)
{
 return head->next;
}

size_t list_length (List *head)
{
 size_t n;
  
 for (n = 0; head; ++n)
  head = head->next;

 return n;
}
\end{Verbatim}

\subsection{Library Implementation}


In order to \underline{set the stage for} ({\MbQ\cH209}... {\MaQ\cH84}\z{\MaQ\cH223}\z{\MbQ\cH196}\z{\MaQ\cH101}) later chapter which 
expand upon this example, in this subsection I will describe the purpose of 
the sources that combine to implement the shell library. I will not 
dissect ({\MdQ\cH29}\z{\McQ\cH30}\z{\MaQ\cH125}\z{\MgQ\cH38}; {\McQ\cH30}\z{\MbQ\cH45}\z{\MbQ\cH251}\z{\McQ\cH10}) the code introduced here -- you can download the 
sources from the book's webpages at http://sources.redhat.com/autobook/. 


The remaining sources for the library, beyond the support files described in the previous subsection, are divided into four pairs of files: 

\subsubsection{`sic.c' \& `sic.h'}

Here are the functions for creating and managing sic parsers. 

\begin{Verbatim}[frame=single]
#ifndef SIC_SIC_H
#define SIC_SIC_H 1

#include <sic/common.h>
#include <sic/error.h>
#include <sic/list.h>
#include <sic/syntax.h>

typedef struct sic 
{
 char *result;         /* result string */
 size_t len;           /* bytes used by result field */
 size_t lim;           /* bytes allocated to result field */
 /* tables of builtin functions */
 struct builtintab *builtins;
 /* dispatch table for syntax of input */
 SyntaxTable **syntax;
 List *syntax_init;  /* stack of syntax state initialisers */
 List *syntax_finish;/* stack of syntax state finalizers */
 SicState *state;    /* state data from syntax extensions */
} Sic;

#endif /* !SIC_SIC_H */
\end{Verbatim}

This structure has fields to store registered command (builtins) and 
syntax (syntax) handlers, along with other state information (state) that can 
be used to share information between various handlers, and some room to build 
a result or error string (result). 

\subsubsection{`builtin.c' \& `builtin.h'}

Here are the functions for managing tables of builtin commands in each Sic structure: 
\begin{Verbatim}[frame=single]
typedef int (*builtin_handler) (Sic *sic,
                                int argc, char *const argv[]);

typedef struct 
{
 const char *name;
 builtin_handler func;
 int min, max;
} Builtin;

typedef struct builtintab BuiltinTab;

extern Builtin *builtin_find (Sic *sic, const char *name);
extern int builtin_install   (Sic *sic, Builtin *table);
extern int builtin_remove    (Sic *sic, Builtin *table);
\end{Verbatim}

\subsubsection{`eval.c' \& `eval.h'}


Having created a Sic parser, and populated ({\MbQ\cH2}\z{\MaQ\cH82}\z{\MbQ\cH107}) it with some Builtin 
handlers, a user of this library must tokenize and evaluate its input stream.
These files define a structure for storing tokenized strings (Tokens), and 
functions for converting char * strings both to and from this structure type: 

\begin{Verbatim}[frame=single]
#ifndef SIC_EVAL_H
#define SIC_EVAL_H 1

#include <sic/common.h>
#include <sic/sic.h>

BEGIN_C_DECLS

typedef struct 
{
 int  argc;            /* number of elements in ARGV */
 char **argv;          /* array of pointers to elements */
 size_t lim;           /* number of bytes allocated */
} Tokens;

extern int eval       (Sic *sic, Tokens *tokens);

extern int 
untokenize (Sic *sic, char **pcommand, Tokens *tokens);

extern int 
tokenize (Sic *sic, Tokens **ptokens, char **pcommand);

END_C_DECLS

#endif /* !SIC_EVAL_H */
\end{Verbatim}

These files also define the eval function, which examines a Tokens structure 
in the context of the given Sic parser, dispatching the argv array to a 
relevant Builtin handler, also written by the library user. 

\subsubsection{`syntax.c' \& `syntax.h'}


When tokenize splits a char * string into parts, by default it breaks the 
string into words delimited by whitespace. These files define the interface 
for changing this default behaviour, by registering callback functions which 
the parser will run when it meets an `interesting' symbol in the input stream.
Here are the declarations from `syntax.h': 

\begin{Verbatim}[frame=single]
BEGIN_C_DECLS

typedef int SyntaxHandler (struct sic *sic, BufferIn *in,
                           BufferOut *out);

typedef struct syntax 
{
 SyntaxHandler *handler;
 char *ch;
}Syntax;

extern int syntax_install(struct sic *sic, Syntax *table);
extern SyntaxHandler *syntax_handler(struct sic *sic, int ch);

END_C_DECLS
\end{Verbatim}

A SyntaxHandler is a function called by tokenize as it consumes its input to 
create a Tokens structure; the two functions associate a table of such handlers
with a given Sic parser, and find the particular handler for a given character 
in that Sic parser, respectively. 

\subsection{Beginnings of a `configure.in'}\label{boac}
Now that I have some code, I can run \textbf{autoscan} to generate a 
preliminary ({\MaQ\cH128}\z{\MbQ\cH165}; {\McQ\cH199}\z{\MhQ\cH171}; {\McQ\cH227}\z{\MaQ\cH101}) `configure.in'. autoscan will examine all of 
the sources in the current directory tree looking for common points of 
non-portability, adding macros suitable for detecting the discovered problems.
\textbf{autoscan} generates the following in `\textbf{configure.scan}': 

\begin{Verbatim}[frame=single]
# Process this file with autoconf to 
# produce a configure script.
AC_INIT(sic/eval.h)

# Checks for programs.

# Checks for libraries.

# Checks for header files.
AC_HEADER_STDC
AC_CHECK_HEADERS(strings.h unistd.h)

#Checks for typedefs,structures,and compiler characteristics.
AC_C_CONST
AC_TYPE_SIZE_T

# Checks for library functions.
AC_FUNC_VPRINTF
AC_CHECK_FUNCS(strerror)

AC_OUTPUT()
\end{Verbatim}

Since the generated `configure.scan' does not overwrite your 
project's `configure.in', it is a good idea to run autoscan 
periodically ({\MaQ\cH236}\z{\MbQ\cH130}\z{\MaQ\cH203}; {\MdQ\cH88}\z{\MhQ\cH21}) even in established project source trees, and 
compare the two files. Sometimes autoscan will find some portability issue 
you have overlooked ({\MbQ\cH245}\z{\MgQ\cH213}; {\MfQ\cH31}\z{\MhQ\cH73}) , or weren't aware of. 

Looking through the documentation for the macros in this `configure.scan',
AC\_{}C\_{}CONST and AC\_{}TYPE\_{}SIZE\_{}T will \underline{take care of}
({\MbQ\cH206}\z{\MjQ\cH232}, {\McQ\cH84}\z{\MbQ\cH220}) themselves (provided ({\MaQ\cH74}... {\MbQ\cH209}\z{\MbQ\cH145}\z{\MaQ\cH75}; {\MaQ\cH99}\z{\MaQ\cH224}) I ensure ({\MaQ\cH92}\z{\McQ\cH132}) 
that `config.h' is included into every source file),
and AC\_{}HEADER\_{}STDC and AC\_{}CHECK\_{}HEADERS (unistd.h) are already 
taken care of in `common.h'. 

autoscan is no silver bullet! Even here in this simple example, I need to 
manually add macros to check for the presence ({\MaQ\cH230}\z{\MaQ\cH202}) of `errno.h': 

\begin{Verbatim}[frame=single]
AC_CHECK_HEADERS(errno.h strings.h unistd.h)
\end{Verbatim}

I also need to manually add the Autoconf macro for generating `config.h'; a 
macro to initialise automake support; and a macro to check for the presence 
of ranlib. These should go \underline{close to} ({\MbQ\cH81}\z{\McQ\cH161}\z{\MbQ\cH107}, {\MaQ\cH202}\z{\McQ\cH203}\z{\McQ\cH161}) the 
start of `configure.in': 

\begin{Verbatim}[frame=single]
...
AC_CONFIG_HEADER(config.h)
AM_INIT_AUTOMAKE(sic, 0.5)

AC_PROG_CC
AC_PROG_RANLIB
...
\end{Verbatim}

Recall that the use of bzero 
in \ref{mm} \textbf{Memory Management} (page \pageref{mm}) is 
not entirely portable. The trick is to provide a bzero work-alike, depending on which functions Autoconf detects, by adding the following towards the end of `configure.in':

\begin{Verbatim}[frame=single]
...
AC_CHECK_FUNCS(bzero memset, break)
...
\end{Verbatim}

With the addition of this small snippet ({\MbQ\cH105}\z{\MbQ\cH210}) of code to `common.h', I can 
now \underline{make use of} ({\MaQ\cH130}\z{\MbQ\cH224}) bzero even when linking with a C library 
that has no implementation of its own: 

\begin{Verbatim}[frame=single]
#if !HAVE_BZERO && HAVE_MEMSET
# define bzero(buf, bytes)  ((void) memset (buf, 0, bytes))
#endif
\end{Verbatim}

An interesting macro suggested by autoscan is AC\_{}CHECK\_{}FUNCS(strerror).
This tells me that I need to provide a replacement implementation of 
strerror for the benefit of architectures which don't have it in their 
system libraries. This is resolved by providing a file with a fallback 
implementation for the named function, and creating a library from it and 
any others that `configure' discovers to be lacking from the 
system library on the target host. 


You will recall that `configure' is the shell script the end user of this 
package will run on their machine to test that it has all the features 
the package wants to use. The library that is created will allow the rest 
of the project to be written in the knowledge that any functions required 
by the project but missing from the installers system libraries will be 
available nonetheless ({\MaQ\cH79}\z{\MbQ\cH117}; {\MaQ\cH69}\z{\MbQ\cH205}) . GNU `libiberty' comes to the 
rescue again -- it already has an implementation of `strerror.c' that I 
was able to use with a little modification. 


Being able to supply a simple implementation of strerror, as the `strerror.c' 
file from `libiberty' does, relies on there being a well 
defined sys\_{}errlist variable. It is a fair bet that if the target host 
has no strerror implementation, however, that the system sys\_{}errlist will 
be broken or missing. I need to write a configure macro to check whether 
the system defines sys\_{}errlist, and tailor ({\MaQ\cH94}\z{\MbQ\cH90}; {\MaQ\cH85}\z{\MaQ\cH175}\z{\MjQ\cH95}) the code 
in `strerror.c' to use this knowledge. 


To avoid clutter ({\MdQ\cH126}\z{\MdQ\cH18}) in the top-level directory, I am a great believer in 
keeping as many of the configuration files as possible in their own 
sub-directory. First of all, I will create a new directory called `config' 
inside the top-level directory, and put `sys\_{}errlist.m4' inside it: 

\begin{Verbatim}[frame=single]
AC_DEFUN(SIC_VAR_SYS_ERRLIST,
[AC_CACHE_CHECK([for sys_errlist],
sic_cv_var_sys_errlist,
[AC_TRY_LINK([int *p;],
  [extern int sys_errlist; p = &sys_errlist;],
  sic_cv_var_sys_errlist=yes, sic_cv_var_sys_errlist=no)])
if test x"$sic_cv_var_sys_errlist" = xyes; then
 AC_DEFINE(HAVE_SYS_ERRLIST, 1,
  [Define if your system libraries have a \
   sys_errlist variable.])
fi])
\end{Verbatim}

I must then add a call to this new macro in the `configure.in' file being careful to put it in the right place -- somwhere between typedefs and structures and library functions according to the comments in `configure.scan': 

\begin{Verbatim}[frame=single]
SIC_VAR_SYS_ERRLIST
\end{Verbatim}

GNU Autotools can also be set to store most of their files in a subdirectory,
by calling the AC\_{}CONFIG\_{}AUX\_{}DIR macro near the top 
of `configure.in', preferably ({\MbQ\cH121}\z{\MaQ\cH170}\z{\MaQ\cH166}\z{\MaQ\cH203}; {\MbQ\cH121}\z{\MaQ\cH223}\z{\MaQ\cH203}; {\MeQ\cH158}\z{\MaQ\cH170}) right after AC\_{}INIT: 

\begin{Verbatim}[frame=single]
AC_INIT(sic/eval.c)
AC_CONFIG_AUX_DIR(config)
AM_CONFIG_HEADER(config.h)
...
\end{Verbatim}

Having made this change, many of the files added by running autoconf and 
automake \verb+--add-missing+ will be put in the aux\_{}dir. 

The source tree now looks like this: 

\begin{Verbatim}[frame=single]
sic/
  +-- configure.scan
  +-- config/
  |     +-- sys_errlist.m4
  +-- replace/
  |     +-- strerror.c
  +-- sic/
        +-- builtin.c
        +-- builtin.h
        +-- common.h
        +-- error.c
        +-- error.h
        +-- eval.c
        +-- eval.h
        +-- list.c
        +-- list.h
        +-- sic.c
        +-- sic.h
        +-- syntax.c
        +-- syntax.h
        +-- xmalloc.c
        +-- xstrdup.c
        +-- xstrerror.c
\end{Verbatim}

In order to correctly utilise the fallback implementation,
AC\_{}CHECK\_{}FUNCS (strerror) needs to be removed and strerror added 
to AC\_{}REPLACE\_{}FUNCS: 

\begin{Verbatim}[frame=single]
# Checks for library functions.
AC_REPLACE_FUNCS(strerror)
\end{Verbatim}

This will be clearer if you look at the `Makefile.am' for the `replace' 
subdirectory: 

\begin{Verbatim}[frame=single]
## Makefile.am -- Process this file with automake
#   to produce Makefile.in

INCLUDES                =  -I$(top_builddir) -I$(top_srcdir)

noinst_LIBRARIES        = libreplace.a
libreplace_a_SOURCES        = 
libreplace_a_LIBADD        = @LIBOBJS@
\end{Verbatim}

The code tells automake that I want to build a library for use within the 
build tree (i.e. not installed -- `noinst'), and that has no source files 
by default. The clever ({\MjQ\cH209}\z{\MeQ\cH205}\z{\MbQ\cH237}) part here is that when someone comes to 
install Sic, they will run configure which will test for strerror, and 
add `strerror.o' to LIBOBJS if the target host environment is missing its 
own implementation. Now, when `configure' creates `replace/Makefile' (as I 
asked it to with AC\_{}OUTPUT), `@LIBOBJS@' is replaced by the list of objects 
required on the installer's machine. 

Having done all this at configure time, when my user runs make, the files 
required to replace functions missing from their target machine will be added 
to `libreplace.a'. 

Unfortunately this is not quite enough to start building the project. First 
I need to add a top-level `Makefile.am' from which to ultimately ({\MbQ\cH124}\z{\MbQ\cH40}; {\McQ\cH31}\z{\MbQ\cH148}\z{\MaQ\cH203})
create a top-level `Makefile' that will descend into the various 
subdirectories of the project: 

\begin{Verbatim}[frame=single]
## Makefile.am -- Process this file with automake to 
#   produce Makefile.in

SUBDIRS = replace sic
\end{Verbatim}

 And `configure.in' must be told where it can find instances of Makefile.in: 

\begin{Verbatim}[frame=single]
AC_OUTPUT(Makefile replace/Makefile sic/Makefile)
\end{Verbatim}

I have written a bootstrap script for Sic, for details 
see \ref{C_Bootstrapping} Bootstrapping (page \pageref{C_Bootstrapping}):

\begin{Verbatim}[frame=single]
#! /bin/sh

set -x
aclocal -I config
autoheader
automake --foreign --add-missing --copy
autoconf
\end{Verbatim}

The `\verb+--foreign+' option to automake tells it to relax ({\MbQ\cH91}\z{\MkQ\cH13}) the GNU 
standards for various files that should be present in a GNU distribution.
Using this option saves me from havng to create empty files as we did 
in \ref{C_amgap} A Minimal GNU Autotools Project (page \pageref{C_amgap}). 


Right. Let's build the library! First, I'll run bootstrap: 
\begin{Verbatim}[frame=single]
$ ./bootstrap
+ aclocal -I config
+ autoheader
+ automake --foreign --add-missing --copy
automake: configure.in: installing config/install-sh
automake: configure.in: installing config/mkinstalldirs
automake: configure.in: installing config/missing
+ autoconf
\end{Verbatim}

The project is now in the same state that an end-user would see, having 
unpacked a distribution tarball. What follows is what an end user might 
expect to see when building from that tarball: 

\begin{Verbatim}[frame=single]
$ ./configure
creating cache ./config.cache
checking for a BSD compatible install... /usr/bin/install -c
checking whether build environment is sane... yes
checking whether make sets ${MAKE}... yes
checking for working aclocal... found
checking for working autoconf... found
checking for working automake... found
checking for working autoheader... found
checking for working makeinfo... found
checking for gcc... gcc
checking whether the C compiler (gcc  ) works... yes
checking whether the C compiler (gcc  ) is a \
 cross-compiler... no
checking whether we are using GNU C... yes
checking whether gcc accepts -g... yes
checking for ranlib... ranlib
checking how to run the C preprocessor... gcc -E
checking for ANSI C header files... yes
checking for unistd.h... yes
checking for errno.h... yes
checking for string.h... yes
checking for working const... yes
checking for size_t... yes
checking for strerror... yes
updating cache ./config.cache
creating ./config.status
creating Makefile
creating replace/Makefile
creating sic/Makefile
creating config.h
\end{Verbatim}

Compare this output with the contents of `configure.in', and notice how each 
macro is ultimately responsible ({\McQ\cH219}\z{\McQ\cH138}\z{\McQ\cH141}\z{\MaQ\cH76}\z{\MbQ\cH237}) for one or more 
consecutive ({\McQ\cH169}\z{\McQ\cH47}\z{\MaQ\cH46}\z{\MbQ\cH105}\z{\MbQ\cH237}) tests (via the Bourne shell code generated 
in `configure'). Now that the `Makefile's have been successfully created,
it is safe to call make to perform the actual compilation: 

\begin{Verbatim}[frame=single]
$ make
make  all-recursive
make[1]: Entering directory `/tmp/sic'
Making all in replace
make[2]: Entering directory `/tmp/sic/replace'
rm -f libreplace.a
ar cru libreplace.a
ranlib libreplace.a
make[2]: Leaving directory `/tmp/sic/replace'
Making all in sic
make[2]: Entering directory `/tmp/sic/sic'
gcc -DHAVE_CONFIG_H -I. -I. -I.. -I..    -g -O2 -c builtin.c
gcc -DHAVE_CONFIG_H -I. -I. -I.. -I..    -g -O2 -c error.c
gcc -DHAVE_CONFIG_H -I. -I. -I.. -I..    -g -O2 -c eval.c
gcc -DHAVE_CONFIG_H -I. -I. -I.. -I..    -g -O2 -c list.c
gcc -DHAVE_CONFIG_H -I. -I. -I.. -I..    -g -O2 -c sic.c
gcc -DHAVE_CONFIG_H -I. -I. -I.. -I..    -g -O2 -c syntax.c
gcc -DHAVE_CONFIG_H -I. -I. -I.. -I..    -g -O2 -c xmalloc.c
gcc -DHAVE_CONFIG_H -I. -I. -I.. -I..    -g -O2 -c xstrdup.c
gcc -DHAVE_CONFIG_H -I. -I. -I.. -I..    -g -O2 -c xstrerror.c
rm -f libsic.a
ar cru libsic.a builtin.o error.o eval.o list.o sic.o \
   syntax.o xmalloc.o xstrdup.o xstrerror.o ranlib libsic.a
make[2]: Leaving directory `/tmp/sic/sic'
make[1]: Leaving directory `/tmp/sic'
\end{Verbatim}

On this machine, as you can see from the output of configure above, I have 
no need of the fallback implementation of strerror, so `libreplace.a' is empty.
On another machine this might not be the case. In any event, I now have a 
compiled `libsic.a' -- so far, so good. 

\section{A Sample Shell Application}

What I need now, is a program that uses `libsic.a', if only to give me 
confidence that it is working. In this section, I will write a simple 
shell which uses the library. But first, I'll create a directory to put it in:

\begin{Verbatim}[frame=single]
$ mkdir src
$ ls -F
COPYING  Makefile.am  aclocal.m4  configure*    config/   sic/
INSTALL  Makefile.in  bootstrap*  configure.in  replace/  src/
$ cd src
\end{Verbatim}

In order to put this shell together, we need to provide just a few 
things for integration with `libsic.a'... 

\subsection{`sic\_{}repl.c'}


In `sic\_{}repl.c'\footnote{Read Eval Print Loop.} there is a loop for reading 
strings typed by the user, evaluating them and printing the results. GNU readline is ideally suited to this, but it is not always available -- or sometimes people simply may not wish to use it. 


With the help of GNU Autotools, it is very easy to cater ({\MjQ\cH67}\z{\MaQ\cH175}) for 
building with and without GNU readline. `sic\_{}repl.c' uses this function 
to read lines of input from the user: 

\begin{Verbatim}[frame=single]
static char * getline (FILE *in, const char *prompt)
{
 /* Always allocated and freed from inside this function.  */
 static char *buf = NULL; 
                           
 XFREE (buf);

 buf = (char *) readline ((char *) prompt);

 #ifdef HAVE_ADD_HISTORY
  if (buf && *buf)
   add_history (buf);
 #endif
  
 return buf;
}
\end{Verbatim}

To make this work, I must write an Autoconf macro which adds an option 
to `configure', so that when the package is installed, it will use the 
readline library if `\verb+--with-readline+' is used: 

\begin{Verbatim}[frame=single]
AC_DEFUN(SIC_WITH_READLINE,
[AC_ARG_WITH(readline,
[ --with-readline  compile with the system readline library],
[if test x"${withval-no}" != xno; then
  sic_save_LIBS=$LIBS
  AC_CHECK_LIB(readline, readline)
  if test x"${ac_cv_lib_readline_readline}" = xno; then
    AC_MSG_ERROR(libreadline not found)
  fi
  LIBS=$sic_save_LIBS
fi])
AM_CONDITIONAL(WITH_READLINE,test x"${with_readline-no}"!=xno)
])
\end{Verbatim}

Having put this macro in the file `config/readline.m4', I must also call the new macro (SIC\_{}WITH\_{}READLINE) from `configure.in'. 

\subsection{`sic\_{}syntax.c'}


The syntax of the commands in the shell I am writing is defined by a set of 
syntax handlers which are loaded into `libsic' at startup. I can get the C 
preprocessor to do most of the repetitive ({\McQ\cH189}\z{\MiQ\cH181}\z{\MbQ\cH237}) code for me, and just fill 
in the function bodies: 

\begin{Verbatim}[frame=single]
#if HAVE_CONFIG_H
#  include <sic/config.h>
#endif

#include "sic.h"

/* List of builtin syntax. */
#define syntax_functions                \
        SYNTAX(escape,  "\\")           \
        SYNTAX(space,   " \f\n\r\t\v")  \
        SYNTAX(comment, "#")            \
        SYNTAX(string,  "\"")           \
        SYNTAX(endcmd,  ";")            \
        SYNTAX(endstr,  "")

/* Prototype Generator. */
#define SIC_SYNTAX(name)                \
        int name (Sic *sic, BufferIn *in, BufferOut *out)

#define SYNTAX(name, string)            \
        extern SIC_SYNTAX (CONC (syntax_, name));
syntax_functions
#undef SYNTAX

/* Syntax handler mappings. */
Syntax syntax_table[] = {

#define SYNTAX(name, string)            \
        { CONC (syntax_, name), string },
  syntax_functions
#undef SYNTAX
  
  { NULL, NULL }
};
\end{Verbatim}

This code writes the prototypes for the syntax handler functions, and creates 
a table which associates each with one or more characters that might occur in 
the input stream. The advantage of writing the code this way is that when I 
want to add a new syntax handler later, it is a simple matter of adding a new 
row to the syntax\_{}functions macro, and writing the function itself. 

\subsection{`sic\_{}builtin.c'}


In addition to the syntax handlers I have just added to the Sic shell, the 
language of this shell is also defined by the builtin commands it provides.
The infrastructure for this file is built from a table of functions which is 
fed ({\MuQ\cH3}) into various C preprocessor macros, just as I did for the syntax 
handlers. 


One builtin handler function has special status, builtin\_{}unknown.
This is the builtin that is called, if the Sic library cannot find a 
suitable builtin function to handle the current input command. At first this 
doesn't sound especially important -- but it is the key to any shell 
implementation. When there is no builtin handler for the command, the shell 
will search the users command path, `\$PATH', to find a suitable executable.
And this is the job of builtin\_{}unknown: 

\begin{Verbatim}[frame=single]
int builtin_unknown (Sic *sic, int argc, char *const argv[])
{
  char *path = path_find (argv[0]);
  int status = SIC_ERROR;

  if (!path)
    {
      sic_result_append (sic, "command \"");
      sic_result_append (sic, argv[0]);
      sic_result_append (sic, "\" not found");
    }
  else if (path_execute (sic, path, argv) != SIC_OKAY)
    {
      sic_result_append (sic, "command \"");
      sic_result_append (sic, argv[0]);
      sic_result_append (sic, "\" failed: ");
      sic_result_append (sic, strerror (errno));
    }
  else
    status = SIC_OKAY;

  return status;
}

static char *path_find (const char *command)
{
 char *path = xstrdup (command);
  
 if (*command == '/')
 {
  if (access (command, X_OK) < 0)
   goto notfound;
 }
 else
 {
  char *PATH = getenv ("PATH");
  char *pbeg, *pend;
  size_t len;

  for (pbeg = PATH; *pbeg != '\0'; pbeg = pend)
  {
   pbeg += strspn (pbeg, ":");
   len = strcspn (pbeg, ":");
   pend = pbeg + len;
   path = XREALLOC (char, path, 2 + len + strlen(command));
   *path = '\0';
   strncat (path, pbeg, len);
   if (path[len -1] != '/') strcat (path, "/");
    strcat (path, command);
          
   if (access (path, X_OK) == 0)
    break;
  }

  if (*pbeg == '\0')
   goto notfound;
 }

 return path;

 notfound:
  XFREE (path);
  return NULL;
}  
\end{Verbatim}

Running `autoscan' again at this point adds AC\_{}CHECK\_{}FUNCS (strcspn
strspn) to `configure.scan'. This tells me that these functions are not 
truly portable. As before I provide fallback implementations for these 
functions incase ({\MeQ\cH12}\z{\MaQ\cH82}) they are missing from the target host -- and as it 
turns out, they are easy to write: 

\begin{Verbatim}[frame=single]
/* strcspn.c -- implement strcspn() 
    for architectures without it */

#if HAVE_CONFIG_H
#  include <sic/config.h>
#endif

#include <sys/types.h>

#if STDC_HEADERS
#  include <string.h>
#elif HAVE_STRINGS_H
#  include <strings.h>
#endif

#if !HAVE_STRCHR
#  ifndef strchr
#    define strchr index
#  endif
#endif

size_t strcspn (const char *string, const char *reject)
{
  size_t count = 0;
  while (strchr (reject, *string) == 0)
    ++count, ++string;

  return count;
}
\end{Verbatim}

There is no need to add any code to `Makefile.am', because the configure script
will automatically add the names of the missing function sources to `@LIBOBJS@'. 


This implementation uses the autoconf generated `config.h' to get information about the availability of headers and type definitions. It is interesting that autoscan reports that strchr and strrchr, which are used in the fallback implementations of strcspn and strspn respectively, are themselves not portable! Luckily, the Autoconf manual tells me exactly how to deal with this: by adding some code to my `common.h' (paraphrased ({\MbQ\cH90}\z{\MeQ\cH159}) from the literal code in the manual): 

\begin{Verbatim}[frame=single]
#if !STDC_HEADERS
#  if !HAVE_STRCHR
#    define strchr index
#    define strrchr rindex
#  endif
#endif
\end{Verbatim}

And another macro in `configure.in': 

\begin{Verbatim}[frame=single]
AC_CHECK_FUNCS(strchr strrchr)
\end{Verbatim}

\subsection{`sic.c' \& `sic.h'}


Since the application binary has no installed header files, there is little point in maintaining a corresponding header file for every source, all of the structures shared by these files, and non-static functions in these files are declared in `sic.h': 

\begin{Verbatim}[frame=single]
#ifndef SIC_H
#define SIC_H 1

#include <sic/common.h>
#include <sic/sic.h>
#include <sic/builtin.h>

BEGIN_C_DECLS

extern Syntax syntax_table[];
extern Builtin builtin_table[];
extern Syntax syntax_table[];

extern int evalstream    (Sic *sic, FILE *stream);
extern int evalline      (Sic *sic, char **pline);
extern int source        (Sic *sic, const char *path);
extern int syntax_init   (Sic *sic);
extern 
 int syntax_finish (Sic *sic, BufferIn *in, BufferOut *out);

END_C_DECLS

#endif /* !SIC_H */
\end{Verbatim}

To hold together everything you have seen so far, the main function creates 
a Sic parser and initialises it by adding syntax handler functions and builtin 
functions from the two tables defined earlier, before handing control to 
evalstream which will eventually ({\MbQ\cH124}\z{\MbQ\cH40}) exit when the input stream is 
exhausted ({\MbQ\cH224}\z{\MaQ\cH234}).

\begin{Verbatim}[frame=single]
int main (int argc, char * const argv[])
{
  int result = EXIT_SUCCESS;
  Sic *sic = sic_new ();
  
  /* initialise the system */
  if (sic_init (sic) != SIC_OKAY)
      sic_fatal ("sic initialisation failed");
  signal (SIGINT, SIG_IGN);
  setbuf (stdout, NULL);

  /* initial symbols */
  sicstate_set (sic, "PS1", "] ", NULL);
  sicstate_set (sic, "PS2", "- ", NULL);
  
  /* evaluate the input stream */
  evalstream (sic, stdin);

  exit (result);
}
\end{Verbatim}

Now, the shell can be built and used: 

\begin{Verbatim}[frame=single]
$ bootstrap
...
$ ./configure --with-readline
...
$ make
...
make[2]: Entering directory `/tmp/sic/src'
gcc -DHAVE_CONFIG_H -I. -I.. -I../sic -I.. \
 -I../sic -g -c sic.c
gcc -DHAVE_CONFIG_H -I. -I.. -I../sic -I.. \
 -I../sic -g -c sic_builtin.c
gcc -DHAVE_CONFIG_H -I. -I.. -I../sic -I.. \
 -I../sic -g -c sic_repl.c
gcc -DHAVE_CONFIG_H -I. -I.. -I../sic -I.. \
 -I../sic -g -c sic_syntax.c
gcc  -g -O2  -o sic  sic.o sic_builtin.o   \
 sic_repl.o sic_syntax.o ../sic/libsic.a   \
 ../replace/libreplace.a -lreadline
make[2]: Leaving directory `/tmp/sic/src'
...
$ ./src/sic
] pwd
/tmp/sic
] ls -F
Makefile     aclocal.m4   config.cache    configure*    sic/
Makefile.am  bootstrap*   config.log      configure.in  src/
Makefile.in  config/      config.status*  replace/
] exit
$
\end{Verbatim}

This chapter has developed a solid foundation of code, which I will return to 
in \ref{C_A_Large_GNU_Autotools_Project} A Large GNU Autotools Project (page
\pageref{C_A_Large_GNU_Autotools_Project}), when Libtool will join the fray.
The chapters leading up to that explain what Libtool is for, how to use it 
and integrate it into your own projects, and the advantages it offers over 
building shared libraries with Automake (or even just Make) alone.

