\chapter{Bootstrapping}\label{C_Bootstrapping}


There are many programs in the GNU Autotools, each of which has a complex set 
of inputs. When one of these inputs changes, it is important to run the 
proper programs in the proper order. Unfortunately, it is hard to remember 
both the dependencies and the ordering. 

For instance, whenever you edit `configure.in', you must remember to re-run 
{\it aclocal} \underline{in case} ({\MaQ\cH99}\z{\MaQ\cH85}, {\MaQ\cH110}\z{\MbQ\cH41}) you 
added a reference to a new macro. You must also rebuild `configure' 
by running {\it autoconf}; `config.h' by running {\it autoheader},
in case you added a new AC\_{}DEFINE; and {\it automake}
to propagate ({\MhQ\cH245}\z{\MgQ\cH122}, {\MaQ\cH211}\z{\MgQ\cH122}) any new AC\_{}SUBSTs to the various `Makefile.in's.
If you edit a `Makefile.am', you must re-run {\it automake}.
In both these cases, you must then remember 
to re-run {\it config.status} -- recheck if `configure' changed, followed 
by config.status to rebuild the `Makefile's. 


When doing active development on the build system for your project,
these dependencies quickly become painful. Of course,
Automake \marginpar{Automake {\MbQ\cH248}\z{\McQ\cH173}\z{\MaQ\cH224}\z{\MaQ\cH83}\z{\MaQ\cH160}\z{\McQ\cH84}\z{\MbQ\cH220}\z{\MbQ\cH243}\z{\MaQ\cH89}\z{\MbQ\cH52}\z{\MbQ\cH237}\z{\MaQ\cH189}\z{\McQ\cH230}{\MaQ\cH1}}
knows how to handle this automatically.
By default, automake generates a `Makefile.in' 
which knows all these dependencies and which automatically re-runs the 
appropriate tools in the appropriate order. These rules assume that the 
correct versions of the tools are all in your PATH. 


It helps to have a script ready to do all of this for you once, before you 
have generated a `Makefile' that will automatically run the tools in 
the correct order, or when you make a fresh ({\MbQ\cH104}\z{\MkQ\cH24}\z{\MbQ\cH237}; {\MbQ\cH132}\z{\McQ\cH37}\z{\MaQ\cH139}\z{\MbQ\cH9}\z{\McQ\cH84}\z{\MbQ\cH220}\z{\McQ\cH172}\z{\MbQ\cH237})
checkout of the code from a CVS repository where the developers don't 
keep generated files under source control. There are at least two 
opposing ({\MaQ\cH250}\z{\McQ\cH222}\z{\MbQ\cH237}; {\MbQ\cH243}\z{\MaQ\cH250}\z{\MbQ\cH237}) schools ({\MaQ\cH231}\z{\MbQ\cH187}, {\MbQ\cH188}\z{\MbQ\cH187}) of 
thought regarding ({\McQ\cH201}\z{\MbQ\cH107}; {\MaQ\cH255}... {\McQ\cH55}\z{\McQ\cH127}) how to go about 
this -- the \underline{autogen.sh school}
and the \underline{bootstrap school}: 

\begin{description}
\item[autogen.sh]
\ %

\underline{From the outset} ({\McQ\cH199}\z{\MaQ\cH225}), this is a poor ({\MiQ\cH246}\z{\MdQ\cH12}\z{\MbQ\cH237}, {\MhQ\cH255}\z{\MaQ\cH253}\z{\MbQ\cH237}, {\MaQ\cH46}\z{\MaQ\cH107}\z{\McQ\cH151}\z{\MbQ\cH237})
name for a bootstrap script,
since there is already a GNU automatic text generation tool called AutoGen.
Often packages that follow this convention have the script automatically 
run the generated configure script after the boostrap process, passing 
autogen.sh arguments through to configure. Except you don't know what options 
you want yet, since you can't run `configure \verb+--+help' until configure 
has been generated. I suggest that if you find yourself compiling a project 
\underline{set up} ({\MoQ\cH42}\z{\McQ\cH13}; {\MbQ\cH29}\z{\McQ\cH168}, {\MbQ\cH29}\z{\McQ\cH13}, {\MdQ\cH147}\z{\McQ\cH13}) in this way that you type:

\begin{Verbatim}[frame=single]
$ /bin/sh ./autogen.sh --help
\end{Verbatim}

and ignore the spurious ({\MaQ\cH99}\z{\MbQ\cH237}; {\MdQ\cH102}\z{\McQ\cH168}\z{\MbQ\cH237}; {\MgQ\cH109}\z{\MkQ\cH4}\z{\MbQ\cH52}\z{\MbQ\cH237}) warning that tells you 
configure will be executed. 

\item[bootstrap]
\ %

Increasingly, projects are starting to call their bootstrap scripts `bootstrap'. Such scripts simply run the various commands required to bring the source tree into a state where the end user can simply:

\begin{Verbatim}[frame=single]
$ configure
$ make
$ make install
\end{Verbatim}

Unfortunately, proponents ({\MbQ\cH84}\z{\McQ\cH135}\z{\MaQ\cH65}; {\MfQ\cH203}\z{\McQ\cH136}\z{\McQ\cH54}) of this school of thought don't put 
the bootstrap script in their distributed tarballs, since the script is 
unnecessary except when the build environment of a developer's machine 
has changed. This means the proponents of the autogen.sh school may 
never see the advantages of the other method. 
\end{description}

Autoconf comes with a program called autoreconf which essentially does the 
work of the bootstrap script. autoreconf is rarely used because,
historically, has not been very well known, and only in Autoconf 2.13 did 
it acquire ({\MaQ\cH166}\z{\MbQ\cH41}, {\MbQ\cH215}\z{\MbQ\cH41}, {\MaQ\cH231}\z{\MaQ\cH131}; {\MjQ\cH244}\z{\MbQ\cH65}) the ability to work with Automake.
Unfortunately, even the Autoconf 2.13 autoreconf does not handle 
libtoolize and some automake-related options that are frequently nice to use. 


We recommend the bootstrap method, until autoreconf is fixed. At this point 
bootstrap has not been standardized, so here is a version of the script we 
used while writing this book\footnote{This book is built using automake and autoconf. We couldn't find a use for libtool.}: 


\begin{Verbatim}[frame=single]
#! /bin/sh

aclocal \
&& automake --gnu --add-missing \
&& autoconf
\end{Verbatim}

We don't use autoreconf here because that script (as of Autoconf 2.13) also 
does not handle the `\verb+--+add-missing' option, which we want. A typical 
bootstrap might also run libtoolize or autoheader. 


It is also important for all developers on a project to have the same
versions of the tools installed so that these rules don't 
inadvertantly ({\MaQ\cH46}\z{\MbQ\cH183}\z{\MbQ\cH60}; {\MlQ\cH80}\z{\MfQ\cH77}; {\MhQ\cH78}\z{\MfQ\cH31}) cause problems due to differences 
between tool versions. This version skew ({\MfQ\cH233}\z{\MbQ\cH237}, {\MmQ\cH11}\z{\MbQ\cH237}, {\MdQ\cH82}\z{\MbQ\cH237}) problem 
\underline{turns out to be}
( ({\MbQ\cH40}\z{\MbQ\cH81}\z{\MaQ\cH177}\z{\MiQ\cH200}\z{\MbQ\cH67}\z{\MbQ\cH36}\z{\MaQ\cH241}\z{\MiQ\cH200}) {\McQ\cH33}\z{\MbQ\cH139}\z{\MbQ\cH117}...; {\MaQ\cH159}\z{\MaQ\cH86}\z{\MbQ\cH117}...; {\McQ\cH132}\z{\MbQ\cH112}\z{\MbQ\cH117}...) fairly 
significant ({\McQ\cH189}\z{\McQ\cH98}\z{\MbQ\cH237}, {\McQ\cH189}\z{\MaQ\cH215}\z{\MbQ\cH237}; {\MdQ\cH77}\z{\MbQ\cH41}\z{\MbQ\cH183}\z{\MbQ\cH60}\z{\MbQ\cH237}) in the field. So,
automake provides a way to disable these 
rules by default, while still allowing users to enable them when they 
know their environment is set up correctly. 


In order to enable this mode, you must first add AM\_{}MAINTAINER\_{}MODE to
 `configure.in'. This will add the `\verb+--+enable-maintainer-mode' option to `configure'; when specified this flag will cause these so-called `maintainer rules' to be enabled. 


Note that maintainer mode is a controversial ({\MhQ\cH17}\z{\McQ\cH127}\z{\MbQ\cH237}; {\MaQ\cH170}\z{\MhQ\cH79}\z{\MbQ\cH237}) feature.
Some people like to use it because it causes fewer bug reports in some 
situations. For instance, CVS does not preserve relative timestamps on files.
If your project has both `configure.in' and `configure' checked in, and 
maintainer mode is not in use, then sometimes make will decide to 
rebuild `configure' even though it is not really required.
This \underline{in turn} ({\MaQ\cH89}\z{\MbQ\cH159}; {\MjQ\cH51}\z{\MaQ\cH131}) means more 
headaches ({\MaQ\cH73}\z{\MaQ\cH65}\z{\McQ\cH229}\z{\MhQ\cH85}\z{\MbQ\cH237}\z{\MaQ\cH57}, {\MkQ\cH44}\z{\MhQ\cH4}) for your 
developers -- on a large project most developers won't touch `configure.in' 
and many may not even want to install 
the GNU Autotools\footnote{Shock ({\MjQ\cH203}\z{\MkQ\cH8}; {\MeQ\cH249}\z{\McQ\cH150}\z{\MjQ\cH203}\z{\MkQ\cH8}\z{\MbQ\cH237}\z{\MaQ\cH57}\z{\MaQ\cH75}),
horror ({\MfQ\cH39}\z{\MfQ\cH35}, {\MjQ\cH203}\z{\MkQ\cH8}, {\MgQ\cH132}\z{\MkQ\cH10}\z{\MlQ\cH92}\z{\MbQ\cH205}) }. 


The other camp ({\MjQ\cH179}\z{\MbQ\cH208}; {\MfQ\cH203}\z{\McQ\cH136}\z{\MgQ\cH47}\z{\McQ\cH248}\z{\MaQ\cH51}\z{\McQ\cH50}) claims ({\MaQ\cH51}\z{\MbQ\cH34}, {\MbQ\cH105}\z{\McQ\cH107}, {\McQ\cH59}\z{\McQ\cH8}) that 
end users should use the same build system that 
developers use, that maintainer mode is simply 
unaesthetic( aesthetic {\MaQ\cH246}\z{\McQ\cH49}\z{\MbQ\cH237}; {\MaQ\cH120}\z{\MbQ\cH127}\z{\MaQ\cH246}\z{\McQ\cH49}\z{\MjQ\cH24}\z{\MdQ\cH226}),
and furthermore ({\McQ\cH55}\z{\MaQ\cH47}, {\MbQ\cH164}\z{\MaQ\cH213}, {\MaQ\cH121}\z{\McQ\cH54}) that the modality ({\MbQ\cH36}\z{\MbQ\cH31}) of 
maintainer mode is dangerous -- you can easily forget what mode you are 
in and thus forget to rebuild, and thus correctly test, a change to 
the configure or build system. When maintainer mode is not in use,
the Automake-supplied missing script will be used to warn users 
when it appears that they need a maintainer tool that they do not have. 


The approach you take depends strongly on the 
social ({\MbQ\cH254}\z{\MaQ\cH62}\z{\MbQ\cH237}, {\MaQ\cH62}\z{\McQ\cH210}\z{\MbQ\cH237}, {\McQ\cH58}\z{\MiQ\cH215}\z{\MbQ\cH237}) structures surrounding your project. 

