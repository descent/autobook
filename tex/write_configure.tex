\chapter{Writing `configure.in'}


Writing a portable `configure.in' is a tricky business. Since you can put 
arbitrary ({\McQ\cH211}\z{\MbQ\cH45}\z{\MbQ\cH70}\z{\MgQ\cH107}\z{\MbQ\cH237}, {\MgQ\cH115}\z{\MbQ\cH105}\z{\MbQ\cH237}) shell code into `configure.in', your options 
seem overwhelming ({\MeQ\cH57}\z{\MdQ\cH74}\z{\MbQ\cH237}; {\MaQ\cH145}\z{\MaQ\cH46}\z{\MaQ\cH170}\z{\MlQ\cH176}\z{\MbQ\cH237}). There are many questions the first-time Autoconf user asks: What constructs are portable and what constructs aren't portable? How do I decide what to check for? What shouldn't I check for? How do I best use Autoconf's features? What shouldn't I put in `configure.in'? In what order should I run my checks? When should I look at the name of the system instead of checking for specific features? 

\section{What is Portability?}


Before we talk about the mechanics ({\MfQ\cH120}\z{\McQ\cH88}\z{\MbQ\cH52}\z{\MbQ\cH237}\z{\McQ\cH182}\z{\MaQ\cH125}; {\MfQ\cH120}\z{\McQ\cH88}; {\MfQ\cH120}\z{\MeQ\cH205}, {\MaQ\cH137}\z{\MaQ\cH231}; {\MbQ\cH156}\z{\MgQ\cH72}\z{\MaQ\cH231}) of
deciding what to check for and how to check for it, let's ask ourselves a simple question: what is portability? Portability is a quality of the code that enables it to be built and run on a variety of (a variety of {\McQ\cH7}\z{\McQ\cH7}) platforms. In the Autoconf context, portability usually refers to the ability to run on Unix-like systems--sometimes including Windows. 


When I first started using Autoconf, I had a hard time deciding what to check for in my `configure.in'. At the time, I was maintaining a
proprietary ({\MhQ\cH146}\z{\MbQ\cH127}\z{\MbQ\cH237}, {\MhQ\cH146}\z{\MbQ\cH208}\z{\MbQ\cH237}; {\McQ\cH139}\z{\MbQ\cH223}\z{\MbQ\cH237}; {\McQ\cH139}\z{\MbQ\cH223}\z{\MbQ\cH158}\z{\MbQ\cH237}; {\McQ\cH139}\z{\MbQ\cH223}\z{\MbQ\cH237}; {\McQ\cH139}\z{\MbQ\cH223}\z{\MbQ\cH158}\z{\MbQ\cH237})
program that ran only on SunOS 4. However, I was interested in porting it to 
Solaris, OSF/1, and possibly Irix. 


The approach I took, while workable, was relatively ({\MbQ\cH243}\z{\MaQ\cH250}\z{\MaQ\cH203}, {\MbQ\cH170}\z{\McQ\cH156}\z{\McQ\cH55}\z{\McQ\cH107}; {\MbQ\cH243}\z{\MbQ\cH231})
time-consuming ({\McQ\cH145}\z{\MbQ\cH118}\z{\MbQ\cH237}; {\MlQ\cH206}\z{\MbQ\cH110}\z{\MbQ\cH77}\z{\MdQ\cH10}\z{\MbQ\cH237}) and painful: I wrote a minimal `configure.in' and then proceeded to simply try to build my program on Solaris. Each time I encountered a build problem, I updated `configure.in' and my source and started again. Once it built correctly, I started testing to see if there were runtime problems related to portability. 


Since I didn't start with a relatively portable base, and since I was
unaware of (unaware of {\MbQ\cH179}\z{\MaQ\cH244}\z{\McQ\cH103}\z{\MaQ\cH131}, {\MbQ\cH179}\z{\MbQ\cH183}\z{\MbQ\cH60}) the tools available to help with adding Autoconf support to a package (see section 24. Migrating an Existing Package to GNU Autotools), it was much more difficult than it had to be. If at all possible, it is better to write portable code to begin with. 


There are a large number of Unix-like systems in the world, including many 
systems which, while still running, can only be considered obsolete. While it 
is probably possible to port some programs to all such systems, typically it 
isn't useful to even try. Porting to everything is a difficult process,
especially given that it usually isn't possible to test on all platforms, and 
that new operating systems, with their own bugs and 
idiosyncracies ( ({\MaQ\cH95}\z{\MaQ\cH65}\z{\MbQ\cH237}) {\MbQ\cH173}\z{\McQ\cH148}, {\McQ\cH51}\z{\MbQ\cH52}) are released every year. 


We advocate ({\MfQ\cH203}\z{\McQ\cH136}; {\MbQ\cH84}\z{\MdQ\cH76}; {\MaQ\cH51}\z{\MbQ\cH34}) a pragmatic ({\MaQ\cH142}\z{\MaQ\cH245}\z{\MbQ\cH237}; {\MaQ\cH245}\z{\MeQ\cH225}\z{\MbQ\cH237}) 
approach ({\MbQ\cH106}\z{\MbQ\cH182}, {\McQ\cH198}\z{\MfQ\cH16}; {\MfQ\cH71}\z{\MbQ\cH25}) to portability: we write our programs to target a 
fairly large, but also fairly modern, cross-section of Unix-like systems. As 
deficiencies ({\MhQ\cH255}\z{\MjQ\cH184}, {\MhQ\cH255}\z{\McQ\cH245}) are discovered in our portability framework, we 
update `configure.in' and our sources, and move on. In practice, this is an 
effective approach.



\section{Brief introduction to portable sh}


If you read a number of `configure.in's, you'll quickly notice that they tend 
to be written in an unusual style. For instance, you'll notice you 
\underline{hardly ever} ({\MbQ\cH39}\z{\MaQ\cH253}\z{\MbQ\cH127})\marginpar{ {\MaQ\cH202} sh script {\MaQ\cH85}\z{\MbQ\cH224} test {\MbQ\cH170}\z{\MaQ\cH85}\z{\MbQ\cH224} [
{\McQ\cH180}\z{\MjQ\cH228}\z{\MhQ\cH245}{\MaQ\cH1}}
see the `[' program used; instead you'll see `test' invoked. We won't go 
into all the details of writing a portable shell script here; instead we leave 
that for chapter \ref{C_Writing_Portable_Bourne_Shell} Writing Portable Bourne 
Shell, page \pageref{C_Writing_Portable_Bourne_Shell}. 


Like other aspects of portability, the approach you take to writing shell scripts in `configure.in' and `Makefile.am' should depend on your goals. Some platforms have notoriously ({\MbQ\cH58}\z{\MaQ\cH177}\z{\MfQ\cH249}\z{\MfQ\cH8}\z{\MaQ\cH203}, {\McQ\cH59}\z{\MaQ\cH177}\z{\MmQ\cH143}\z{\MiQ\cH142}\z{\MaQ\cH203}) broken sh implementations. For instance, Ultrix sh doesn't implement unset. Of course, the GNU Autotools are written in the most portable style possible, so as not to limit your possibilities. 


Also, it doesn't really \underline{make sense} ({\MbQ\cH127}\z{\MbQ\cH60}\z{\McQ\cH50})  to talk about 
portable sh programming in the abstract. sh by itself does very little; most 
actual work is done by separate programs, each with its own potential 
portability problems. For instance, some options are not portable between 
systems, and some seemingly ({\McQ\cH91}\z{\McQ\cH222}\z{\MaQ\cH44}; {\MaQ\cH78}\z{\MdQ\cH11}\z{\MbQ\cH117}) common programs don't exist on 
every system -- so not only do you have to know which sh constructs are not 
portable, but you also must know which programs you can (and cannot) use, and 
which options to those programs are portable. 


This seems daunting ({\MaQ\cH73}\z{\MaQ\cH65}\z{\MlQ\cH82}\z{\MbQ\cH165}\z{\MbQ\cH237}; {\MaQ\cH85}\z{\MaQ\cH65}\z{\MbQ\cH173}\z{\MoQ\cH194}\z{\MbQ\cH237}), but
\underline{in practice} ({\MaQ\cH202}\z{\MaQ\cH245}\z{\MjQ\cH33}\z{\MaQ\cH44}; {\MaQ\cH245}\z{\McQ\cH210}\z{\MaQ\cH44}) it doesn't seem 
to be too hard to write portable shell scripts -- once you've 
internalized ({\MaQ\cH85} ({\McQ\cH51}\z{\MdQ\cH65}{\MaQ\cH2} {\MbQ\cH196}\z{\MaQ\cH134}\z{\McQ\cH16}\z{\McQ\cH37}\z{\MdQ\cH220}\z{\MbQ\cH89}\z{\MaQ\cH176}\z{\MaQ\cH147}\z{\McQ\cH55}) {\MaQ\cH113}\z{\MaQ\cH202}\z{\MaQ\cH147}) the rules. Unfortunately,
this process can take a long time. Meanwhile ({\MaQ\cH119}\z{\MbQ\cH118}, {\MaQ\cH119}\z{\McQ\cH200}) , a 
pragmatic ({\MaQ\cH142}\z{\MaQ\cH245}\z{\MbQ\cH237}; {\MaQ\cH245}\z{\MeQ\cH225}\z{\MbQ\cH237}) `try and see' approach, while noting other portable 
code you've seen elsewhere ({\MaQ\cH202}\z{\MaQ\cH129}\z{\McQ\cH84}; {\MbQ\cH37}\z{\MaQ\cH129}\z{\McQ\cH84}; {\MaQ\cH131}\z{\MaQ\cH129}\z{\McQ\cH84}), works fairly well.
Once again, it pays to \underline{be aware of} ({\MbQ\cH60}\z{\McQ\cH133}\z{\MaQ\cH131}) which architectures 
you'll probably care about -- you will make different choices if you are 
writing an extremely ({\McQ\cH221}\z{\MbQ\cH18}) portable program like emacs or gcc than if you 
are writing something that will only run on various flavors of Linux. Also,
the cost of having unportable code in `configure.in' is relatively low -- in 
general it is fairly easy to rewrite pieces on demand as unportable constructs 
are found. 

\section{Ordering Tests}


In addition to the problem of writing portable sh code, another problem which 
confronts ({\MjQ\cH67}\z{\McQ\cH222}\z{\MjQ\cH88}\z{\MaQ\cH131}; {\McQ\cH222}\z{\McQ\cH64}; {\MjQ\cH96}\z{\MjQ\cH88}) first-time `configure.in' writers is determining the order in which to run the various tests. Autoconf indirectly (via the autoscan program, which we cover in 24. Migrating an Existing Package to GNU Autotools) suggests a standard ordering, which is what we describe here. 


The standard ordering is:
\begin{enumerate}
\item Boilerplate ({\MbQ\cH206}\z{\MbQ\cH133}\z{\MeQ\cH142}\z{\McQ\cH3}\z{\MbQ\cH237}\z{\McQ\cH100}\z{\MhQ\cH184}; ({\MbQ\cH104}\z{\McQ\cH57}\z{\MbQ\cH147}) {\McQ\cH248}\z{\MbQ\cH65}\z{\MaQ\cH46}\z{\McQ\cH137}\z{\MbQ\cH237}\z{\McQ\cH207}\z{\MiQ\cH52}\z{\MbQ\cH100}\z{\MaQ\cH229}) . This section 
should include standard boilerplate code,
such as the call to AC\_{}INIT (which must be first),
AM\_{}INIT\_{}AUTOMAKE, AC\_{}CONFIG\_{}HEADER, and perhaps AC\_{}REVISION. 

\item Options. The next section should include macros which add 
command-line options to configure, such as AC\_{}ARG\_{}ENABLE.
It is typical to put support code for the option in this section as well,
if it is short enough, like this example from libgcj: 

\begin{Verbatim}[frame=single]
AC_ARG_ENABLE(getenv-properties,
[  --disable-getenv-properties
     don't set system properties from GCJ_PROPERTIES])

dnl Whether GCJ_PROPERTIES is used depends on the target.
if test -n "$enable_getenv_properties"; then
 enable_getenv_properties=${
  enable_getenv_properties_default-yes}
fi
if test "$enable_getenv_properties" = no; then
 AC_DEFINE(DISABLE_GETENV_PROPERTIES)
fi
\end{Verbatim}

\item Programs. Next it is traditional to check for programs that are either needed by the configure process, the build process, or by one of the programs being built. This usually involves calls to macros like 
AC\_{}CHECK\_{}PROG and AC\_{}PATH\_{}TOOL. 

\item Libraries. Checks for libraries come before checks for other objects visible to C (or C++, or anything else). This is necessary because some other checks work by trying to link or run a program; by checking for libraries first you ensure that the resulting programs can be linked. 

\item Headers. Next come checks for existence of headers. 

\item Typedefs and structures. We do checks for typedefs after checking for headers for the simple reason that typedefs appear in headers, and we need to know which headers we can use before we look inside them. 

\item Functions. Finally we check for functions. These come last because 
functions have dependencies on the preceding ({\MaQ\cH202}\z{\MaQ\cH135}\z{\MbQ\cH237}, {\MaQ\cH202}\z{\MaQ\cH108}\z{\MbQ\cH237}; {\MaQ\cH135}\z{\McQ\cH222}\z{\MbQ\cH237}) items:
when searching for functions, libraries are needed in order to correctly 
link, headers are needed in order to find prototypes (this is especially 
important for C++, which has stricter prototyping rules than C), and typedefs 
are needed for those functions which use or return types which are not built in. 

\item Output. This is done by invoking AC\_{}OUTPUT. 
\end{enumerate}

This ordering should be considered a rough guideline, and not a list of 
hard-and-fast ({\MaQ\cH46}\z{\MaQ\cH241}\z{\McQ\cH137}\z{\MbQ\cH121}\z{\MbQ\cH237}; {\MaQ\cH195}\z{\MbQ\cH143}\z{\MbQ\cH237}) rules. Sometimes it is necessary to 
interleave ({\MfQ\cH158}\z{\MaQ\cH112}, {\MfQ\cH158}\z{\MdQ\cH197}) tests, either to make `configure.in' easier to 
maintain, or because the tests themselves do need to be in a different order. For instance, if your project uses both C and C++ you might choose to do all the C++ checks after all the C checks are done, in order to make `configure.in' 
\underline{a bit} ({\MbQ\cH127}\z{\McQ\cH245}) easier to read. 

\section{What to check for}


Deciding what to check for is really the central part of 
writing `configure.in'. Once you've read the Autoconf reference manual,
the ``how''s of writing a particular test should be fairly clear.
The ``when''s might remain a mystery -- and it's just as easy to check for 
too many things as it is to check for too few. 


One notable area of divergence ({\MiQ\cH42}\z{\McQ\cH215}; {\MaQ\cH125}\z{\McQ\cH215}; {\MbQ\cH243}\z{\MbQ\cH230}) between various Unix-like 
systems is that the same programs don't exist on all systems, and, even when 
they do, they don't always work in the same way. For these problems we 
recommend, when possible, following the advice of the GNU Coding Standards: 
use the most common options from a relatively limited set of programs. 
Failing ({\MaQ\cH224}\z{\MbQ\cH139}\z{\MbQ\cH179}\z{\MbQ\cH127}...; {\McQ\cH74}\z{\MbQ\cH204}... {\MbQ\cH118}) that, try to stick to ({\MfQ\cH30}\z{\MbQ\cH107}; {\MaQ\cH93}\z{\MaQ\cH232})
programs and options specified by POSIX, perhaps augmenting this approach 
by doing checks for known problems on platforms you care about. 


Checking for tools and their differences is usually a fairly small part of a `configure' script; more common are checks for functions, libraries, and the like. 


\underline{Except for} ({\McQ\cH206}\z{\MaQ\cH55}... {\MaQ\cH74}\z{\MaQ\cH213}, {\McQ\cH98}\z{\MaQ\cH46}\z{\MbQ\cH117}\z{\MbQ\cH226}\z{\MbQ\cH107}) a few core libraries 
like `libc' and, usually, `libm' and libraries like `libX11' which typically 
aren't considered system libraries, there isn't much 
agreement ({\MaQ\cH176}\z{\MbQ\cH60}, {\McQ\cH248}\z{\MbQ\cH95}, {\MaQ\cH155}\z{\MaQ\cH236}, {\MaQ\cH155}\z{\McQ\cH135}) about library 
names or contents between Unix systems. Still, libraries are easy to handle,
because decisions about libraries almost always only affect the 
various `Makefile's. That means that checking for another library typically 
doesn't require major (or even, sometimes, any) changes to the source code.
Also, because adding a new library test has a small impact ({\MfQ\cH9}\z{\McQ\cH224}; {\MaQ\cH84}\z{\MbQ\cH224}) on the 
development cycle -- effectively just re-running `configure' and then a 
relink -- you can effectively ({\MbQ\cH127}\z{\MbQ\cH94}\z{\MaQ\cH203}; {\MbQ\cH222}\z{\MbQ\cH94}\z{\MaQ\cH203}) adopt ({\MbQ\cH83}\z{\MaQ\cH166}; {\MbQ\cH83}\z{\MhQ\cH209}; {\MdQ\cH220}\z{\MbQ\cH89}) a 
lax ({\MaQ\cH46}\z{\MaQ\cH195}\z{\MbQ\cH143}\z{\MbQ\cH237}; {\McQ\cH239}\z{\McQ\cH239}\z{\MiQ\cH150}\z{\MiQ\cH150}\z{\MbQ\cH237}) approach to libraries. For instance, you can just 
make things work on the few systems you immediately care about and then 
handle library changes on an as-needed basis. 


Suppose ({\MmQ\cH147}\z{\MbQ\cH59}, {\MaQ\cH74}\z{\MbQ\cH209}) you do \underline{end up} ({\McQ\cH33}\z{\MgQ\cH33}, {\MbQ\cH167}\z{\MdQ\cH22}) with a link 
problem. How do you handle it? The 
first thing to do is use nm to look through the system libraries to see 
if the missing function exists. If it does, and it is in a library you 
can use then the solution is easy -- just add another AC\_{}CHECK\_{}LIB.
Note that just finding the function in a library is not enough, because on 
some systems, some ``standard'' libraries are 
undesirable ({\MaQ\cH73}\z{\MaQ\cH65}\z{\MaQ\cH46}\z{\MbQ\cH48}\z{\MbQ\cH237}, {\MaQ\cH46}\z{\MaQ\cH167}\z{\MbQ\cH161}\z{\MjQ\cH67}\z{\MbQ\cH237}; {\McQ\cH110}\z{\MdQ\cH192}\z{\MbQ\cH237}); `libucb' is the 
most common example of a library which you should avoid. 

If you can't find the function in a system library then you have a 
somewhat ({\MbQ\cH127}\z{\McQ\cH245}, {\MhQ\cH154}\z{\MfQ\cH22}) more difficult problem: a non-portable function.
There are basically three approaches to a missing function. Below we talk about 
functions, but really these same approaches apply,
\underline{more or less} ({\MaQ\cH214}\z{\MaQ\cH253}\z{\MbQ\cH127}\z{\MaQ\cH60}; {\MaQ\cH215}\z{\McQ\cH26}), to typedefs, structures,
and global variables.  

The first approach is to write a replacement function and either 
conditionally ({\McQ\cH203}\z{\MbQ\cH127}\z{\MbQ\cH145}\z{\MaQ\cH75}\z{\MaQ\cH203}) compile it, or put it into an 
appropriately-named (appropriately {\MjQ\cH95}\z{\MbQ\cH231}\z{\MaQ\cH203}, {\MaQ\cH175}\z{\MjQ\cH95}\z{\MaQ\cH203}, {\MbQ\cH243}\z{\McQ\cH8}\z{\MaQ\cH203}) file and use 
AC\_{}REPLACE\_{}FUNCS. For instance, Tcl uses AC\_{}REPLACE\_{}FUNCS (strstr) 
to handle systems that have no strstr function. 


The second approach is used when there is a similar function with a different 
name. The idea here is to check for all the alternatives and then modify 
your source to use whichever ({\MbQ\cH204}\z{\McQ\cH127}\z{\MkQ\cH168}\z{\MaQ\cH95}; {\MbQ\cH204}\z{\McQ\cH127}\z{\MkQ\cH168}\z{\MaQ\cH60}) one might exist.
The idiom ({\MfQ\cH85}\z{\MbQ\cH224}\z{\McQ\cH120}; {\MbQ\cH65}\z{\McQ\cH120}; {\McQ\cH51}\z{\MfQ\cH85}\z{\McQ\cH120}) here is to use break in the second 
argument to AC\_{}CHECK\_{}FUNCS; this is used both to skip unnecessary tests 
and to indicate to the reader that these checks are related. For instance,
here is how libgcj checks for inet\_{}aton or inet\_{}addr; it only uses 
the first one found:

\begin{Verbatim}[frame=single]
AC_CHECK_FUNCS(inet_aton inet_addr, break)
\end{Verbatim}

Code to use the results of these checks looks something like: 


\begin{Verbatim}[frame=single]
#if HAVE_INET_ATON
  ... use inet_aton here
#else
#if HAVE_INET_ADDR
  ... use inet_addr here
#else
#error Function missing!
#endif
#endif
\end{Verbatim}

Note how we've made it a compile-time error if the function does not exist.
In general it is best to make errors occur as early as possible in the 
build process. 


The third approach to non-portable functions is to write code such that 
these functions are only optionally used. For instance, if you are writing 
an editor you might decide to use mmap to map a file into the editor's memory.
However, since mmap is not portable\marginpar{{\McQ\cH165}\z{\McQ\cH248}\z{\MdQ\cH198}\z{\MaQ\cH228}\z{\MbQ\cH78}\z{\MaQ\cH124} mmap {\MaQ\cH46}\z{\MaQ\cH120}\z{\MaQ\cH170}\z{\MfQ\cH190}\z{\MbQ\cH52}},
you would also write a function to use the more portable read. 


Handling known non-portable functions is only part of the problem, however.
The pragmatic ({\MaQ\cH142}\z{\MaQ\cH245}\z{\MbQ\cH237}; {\MaQ\cH245}\z{\MeQ\cH225}\z{\MbQ\cH237}) approach works fairly well, but it is somewhat 
inefficient ({\MbQ\cH204}\z{\MbQ\cH94}\z{\MhQ\cH48}\z{\MbQ\cH237}; {\MbQ\cH94}\z{\McQ\cH63}\z{\MeQ\cH207}\z{\MbQ\cH237}; {\MbQ\cH204}\z{\McQ\cH63}\z{\MbQ\cH237}, {\MaQ\cH46}\z{\McQ\cH8}\z{\McQ\cH60}\z{\MbQ\cH237}) if you 
are primarily ({\McQ\cH237}\z{\MaQ\cH108}; {\McQ\cH150}\z{\MaQ\cH128}, {\MaQ\cH159}\z{\MaQ\cH86}) developing on a more modern system, like GNU/Linux, which has few functions missing. In this case the problem is that you might not notice non-portable constructs in your code until it has largely been finished. 


Unfortunately, there's no high road to solving this problem. In the end, you 
need to have a working knowledge of the range of existing Unix systems.
Knowledge of standards such as POSIX and XPG can be useful here, as a 
first cut -- if it isn't in POSIX, you should at least consider checking for it.
However, standards are not a panacea ({\McQ\cH78}\z{\McQ\cH63}\z{\MiQ\cH147}; {\McQ\cH94}\z{\MbQ\cH97}\z{\MaQ\cH53}\z{\McQ\cH173}) -- not all systems 
are POSIX compliant,
and sometimes there are bugs in systems functions which you must work around. 


One final class of problems you might encounter is that it is also easy to 
check for too much. This is bad because it adds unnecessary maintenance 
burden ({\McQ\cH189}\z{\McQ\cH138}, {\McQ\cH189}\z{\MbQ\cH86}; {\McQ\cH138}\z{\MbQ\cH86}, {\MgQ\cH143}\z{\McQ\cH189}\z{\MbQ\cH237}\z{\McQ\cH141}\z{\MaQ\cH76}) to your program. For instance, sometimes 
you'll see code that checks for $<$sys/types.h$>$. However, there's no point 
in doing that -- using this header is mostly portable. Again, this can only be 
addressed by having a practical knowledge, which is only really possible by 
examining your target systems. 

\section{Using Configuration Names}

While feature tests are definitely the best approach, a `configure' script may 
occasionally ({\MdQ\cH88}\z{\MhQ\cH21}, {\McQ\cH200}\z{\MbQ\cH67}) have to make a decision based on a configuration 
name. This may be necessary if certain code must be compiled differently 
based on something which can not be tested using a standard Autoconf feature 
test. For instance, the expect ({\McQ\cH227}\z{\MbQ\cH101}; {\McQ\cH227}\z{\MbQ\cH130}) package needs to find 
information about the system's `tty' implementation; this can't 
reliably ({\MaQ\cH170}\z{\MjQ\cH212}\z{\MaQ\cH203}; {\MbQ\cH252}\z{\MaQ\cH245}\z{\MaQ\cH203}) be done when cross compiling without examining 
the particular configuration name.

It is normally better to test for particular features, rather than to test for 
a particular system type. This is because as Unix and other operating 
systems evolve ({\MaQ\cH85}\z{\MjQ\cH77}\z{\MbQ\cH165}\z{\MbQ\cH36}\z{\MbQ\cH65}; {\MbQ\cH234}\z{\MbQ\cH4}, {\MbQ\cH4}\z{\McQ\cH199}) , different systems copy features 
from one another.

When there is no alternative to testing the configuration name in 
a `configure' script, it is best to define a macro which describes the feature,
rather than defining a macro which describes the particular system. This 
permits the same macro to be used on other systems which 
adopt ({\MbQ\cH83}\z{\MaQ\cH166}; {\MbQ\cH83}\z{\MhQ\cH209}; {\MdQ\cH220}\z{\MbQ\cH89}) the 
same feature (see chapter \ref{C_Writing_New_Macros_for_Autoconf}
Writing New Macros for Autoconf,
page \pageref{C_Writing_New_Macros_for_Autoconf}).

Testing\marginpar{test a particular system {\McQ\cH248}\z{\MiQ\cH79}\z{\McQ\cH183}\z{\MbQ\cH125}\z{\MaQ\cH85}\z{\MbQ\cH224} case {\MdQ\cH197}\z{\MjQ\cH72}{\MaQ\cH1}}
for a particular system is normally done using a case statement in 
the autoconf `configure.in' file. The case statement might look something 
like the following, assuming that `host' is a shell variable holding a 
canonical ({\MbQ\cH153}\z{\MbQ\cH196}\z{\MbQ\cH237}; {\MbQ\cH158}\z{\MeQ\cH110}\z{\MbQ\cH237}) configuration system -- which will be the case 
if `configure.in' uses the `AC\_{}CANONICAL\_{}HOST' 
or `AC\_{}CANONICAL\_{}SYSTEM' macros.

 	
\begin{verbatim}
case "${host}" in
i[[3456]]86-*-linux-gnu*) do something ;;
sparc*-sun-solaris2.[[56789]]*) do something ;;
sparc*-sun-solaris*) do something ;;
mips*-*-elf*) do something ;;
esac
\end{verbatim}

Note the doubled square brackets\marginpar{{\MaQ\cH115}\z{\MaQ\cH95}\z{\MaQ\cH50}\z{\MfQ\cH142}\z{\MlQ\cH67}} in this piece of code.
These are used to work around an ugly implementation detail of autoconf --- it
uses M4 under the hood ({\MaQ\cH78}\z{\McQ\cH234}\z{\MeQ\cH216}\z{\MbQ\cH237}\z{\MbQ\cH137}\z{\McQ\cH97}; {\MiQ\cH1}; {\McQ\cH154}\z{\MnQ\cH40}; {\McQ\cH81}\z{\MaQ\cH228}). Without these extra 
brackets, the square brackets in the case statement would be
swallowed ({\MdQ\cH213}\z{\MaQ\cH45}, {\MpQ\cH148}\z{\MaQ\cH45}) by M4, and would not appear in the 
resulting `configure'. This nasty ({\McQ\cH216}\z{\McQ\cH84}\z{\MbQ\cH220}\z{\MbQ\cH237}, {\MaQ\cH189}\z{\McQ\cH230}\z{\MaQ\cH214}\z{\MbQ\cH237}; {\McQ\cH110}\z{\MdQ\cH192}\z{\MbQ\cH237}) detail is 
discussed at more length in \ref{C_M4} M4, page \pageref{C_M4}.

It is particularly important to use `*' after the operating system field, in 
order to match the version number which will be generated by `config.guess'.
In most cases you must be careful to match a range of processor types. For 
most processor families, a trailing `*' suffices, as in `mips*' above. For 
the i386 family, something along ({\MgQ\cH149}\z{\McQ\cH79}; {\MjQ\cH223}\z{\McQ\cH79}) the lines of `i[34567]86' 
suffices ({\McQ\cH151}\z{\MeQ\cH68}) \underline{at present} ({\MbQ\cH241}\z{\MaQ\cH135}; {\MbQ\cH219}\z{\MaQ\cH202}). For the m68k family,
you will need something like `m68*'. Of course, if you do not need to 
match on the processor, it is simpler to just replace the entire field 
by a `*', as in `*-*-irix*'. 

