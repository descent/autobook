\chapter{Introducing GNU Automake}\label{C_Introducing_GNU_Automake}


The primary goal of Automake is to generate `Makefile.in's compliant with the 
GNU Makefile Standards. Along the way, it tries to remove 
boilerplate ({\MbQ\cH206}\z{\MbQ\cH133}\z{\MeQ\cH142}\z{\McQ\cH3}\z{\MbQ\cH237}\z{\McQ\cH100}\z{\MhQ\cH184}; ({\MbQ\cH104}\z{\McQ\cH57}\z{\MbQ\cH147}) {\McQ\cH248}\z{\MbQ\cH65}\z{\MaQ\cH46}\z{\McQ\cH137}\z{\MbQ\cH237}\z{\McQ\cH207}\z{\MiQ\cH52}\z{\MbQ\cH100}\z{\MaQ\cH229}) and 
drudgery ({\McQ\cH75}\z{\MbQ\cH9}; {\MjQ\cH5}\z{\MfQ\cH10}; {\MaQ\cH192}\z{\McQ\cH124}\z{\MgQ\cH143}\z{\MfQ\cH56}\z{\MbQ\cH237}\z{\MbQ\cH9}\z{\MaQ\cH84}). It also helps the `Makefile' writer by 
implementing features (for instance automatic dependency tracking and parallel 
make support) that most maintainers don't have the patience to implement by 
hand. It also implements some best practices ({\McQ\cH51}\z{\MfQ\cH85}, {\MbQ\cH18}\z{\McQ\cH100}, {\MfQ\cH85}\z{\MaQ\cH87}) as well as 
workarounds for 
vendor make bugs -- both of which require 
arcane ({\MbQ\cH255}\z{\MrQ\cH190}\z{\MbQ\cH237}; {\MaQ\cH46}\z{\MaQ\cH170}\z{\MbQ\cH50}\z{\McQ\cH135}\z{\MbQ\cH237}; {\MlQ\cH200}\z{\MqQ\cH248}\z{\McQ\cH216}\z{\McQ\cH106}\z{\MbQ\cH237})
knowledge not generally 
available. 


A secondary goal for Automake is that it work well with other free software, and, specifically, GNU tools. For example, Automake has support for Dejagnu-based test suites. 


Chances are that you don't care about the GNU Coding Standards. That's okay.
You'll still appreciate ({\MgQ\cH106}\z{\MjQ\cH3}, {\MjQ\cH3}\z{\McQ\cH133}, {\MbQ\cH62}\z{\MiQ\cH225}, {\MbQ\cH62}\z{\MgQ\cH233}, {\McQ\cH241}\z{\MbQ\cH125}, {\McQ\cH228}\z{\MbQ\cH125}, {\MaQ\cH244}\z{\MbQ\cH248}) the 
convenience that Automake provides, and you'll find that the GNU standards 
compliance feature, for the most part,
assists rather than impedes ({\MeQ\cH94}\z{\MmQ\cH243}, {\MjQ\cH174}\z{\MmQ\cH243}; {\MjQ\cH174}\z{\MbQ\cH162}). 


Automake helps the maintainer with five large tasks, and countless minor ones. The basic functional areas are: 


\begin{enumerate}
\item Build 
\item Check 
\item Clean 
\item Install and uninstall 
\item Distribution 
\end{enumerate}

We cover the first three items in this chapter, and the others in later 
chapters.\marginpar{{\McQ\cH165}\z{\McQ\cH248}\z{\McQ\cH14}\z{\MdQ\cH201}\z{\McQ\cH110}\z{\McQ\cH127} Build, Check, Clean}
Before we get into the details, let's talk a bit about some 
general principles of Automake. 

\section{General Automake principles}

Automake at its simplest turns ({\MaQ\cH85}\z{\McQ\cH137}\z{\MbQ\cH41}; {\MaQ\cH85}\z{\MbQ\cH65}\z{\MbQ\cH209} [(+into)] ) a file called `Makefile.am' into a 
GNU-compliant `Makefile.in' for use with `configure'. Each `Makefile.am' is 
written according to make syntax; Automake recognizes special macro and target 
names and generates code based on these. 

There are a few Automake rules which differ slightly from make rules: 

\begin{itemize}
\item Ordinary make comments are passed through to the output, but comments 
beginning with `\#\#' are Automake comments and are not passed through. 

\item Automake supports include directives. These directives are not passed 
through to the `Makefile.in', but instead are processed by automake -- files 
included this way are treated as if they were textually ({\MbQ\cH133}\z{\MbQ\cH100}\z{\MaQ\cH203}; {\MaQ\cH159}\z{\MbQ\cH100}\z{\MaQ\cH203})
included 
in `Makefile.am' at that point. This can be used to add boilerplate to 
each `Makefile.am' in a project via a centrally-maintained file. The filename 
to include can start with `\$(top\_{}srcdir)' to indicate that it should be 
found relative to the top-most directory of the project; if it is a 
relative path or if it starts with `\$(srcdir)' then it is relative to 
the current directory. For example, here is how you would reference 
boilerplate code from the file `config/Make-rules' (where `config' is a 
top-level dirctory in the project): 

\begin{Verbatim}[frame=single]
include $(top_srcdir)/config/Make-rules
\end{Verbatim}



\item Automake supports conditionals which are not passed directly through 
to `Makefile.in'. This feature is discussed 
in chapter \ref{C_Advanced_GNU_Automake_Usage} Advanced GNU Automake Usage,
page \pageref{C_Advanced_GNU_Automake_Usage}. 

\item Automake supports macro assignment using `+='; these assignments are translated by Automake into ordinary `=' assignments in `Makefile.in'.

\end{itemize}

All macros and targets, including those which Automake does not recognize, are 
passed through to the generated `Makefile.in' -- this is a powerful extension 
mechanism. Sometimes Automake will define macros or targets internally. If 
these are also defined in `Makefile.am' then the definition 
in `Makefile.am' takes precedence ({\MbQ\cH2}\z{\MaQ\cH135}; {\McQ\cH228}\z{\MaQ\cH108}).
This feature provides an easy way 
to tailor ({\MaQ\cH94}\z{\MbQ\cH90}; {\MaQ\cH85}\z{\MaQ\cH175}\z{\MjQ\cH95}) specific parts of the output in small ways. 


Note, however, that it is a mistake to override parts of the generated code that aren't documented (and thus `exported' by Automake). Overrides like this stand a good chance of not working with future Automake releases. 


Automake also scans `configure.in'. Sometimes it uses the information it discovers to generate extra code, and sometimes to provide extra error checking. Automake also turns every AC\_{}SUBST into a `Makefile' variable. This is convenient in more ways than one: not only does it mean that you can refer to these macros in `Makefile.am' without extra work, but, since Automake scans `configure.in' before it reads any `Makefile.am', it also means that special variables and overrides Automake recognizes can be defined once in `configure.in'. 

\section{Introduction to Primaries}

Each type of object that Automake understands has a special root variable name 
associated with it. \underline{This root is called a primary.} Many actual
variable names put into `Makefile.am' are constructed by adding various 
prefixes to a primary. 


For instance, scripts--interpreted executable programs--are associated with the SCRIPTS primary. Here is how you would list scripts to be installed in the user's `bindir': 

\begin{Verbatim}[frame=single]
bin_SCRIPTS = magic-script
\end{Verbatim}

(Note that the mysterious `bin\_{}' prefix will be discussed later.) 


The contents of a primary-derived variable are treated as targets in the resulting `Makefile'. For instance, in our example above, we could generate `magic-script' using sed by simply introducing it as a target: 

\begin{Verbatim}[frame=single]
bin_SCRIPTS = magic-script

magic-script: magic-script.in
 sed -e 's/whatever//' < $(srcdir)/magic-script.in > 
  magic-script
 chmod +x magic-script
\end{Verbatim}

\section{The easy primaries}


This section describes the common primaries that are relatively easy to understand; the more complicated ones are discussed in the next section. 

\begin{description}
\item [DATA]
\ %

This is the easiest primary to understand. A macro of this type lists a number of files which are installed verbatim. These files can appear either in the source directory or the build directory. 

\item [HEADERS]
\ %

Macros of this type list header files. These are separate from DATA macros because this allows for extra error checking in some cases. 

\item [SCRIPTS]
\ %

This is used for executable scripts (interpreted programs). These are different from DATA because they are installed with different permissions and because they have the program name transform applied to them (e.g., the `--program-transform-name' argument to configure). Scripts are also different from compiled programs because the latter can be stripped while scripts cannot. 

\item [MANS]
\ %

This lists man pages. Installing man pages is more complicated than you might 
think due to the lack of a single common practice. One developer might 
name a man page in the source tree `foo.man' and then rename to the real 
name (`foo.1') at install time. Another developer might instead use numeric 
suffixes in the source tree and install using the same name. Sometimes an 
alphabetic code follows the numeric suffix (e.g., `quux.3n'); this code 
must be stripped before determining the correct install directory (this file 
must still be installed in `\$(man3dir)'). Automake supports all of these 
modes of operation:

\begin{itemize}
\item man\_{}MANS can be used when numeric suffixes are already in place: 

\begin{Verbatim}[frame=single]
man_MANS = foo.1 bar.2 quux.3n
\end{Verbatim}

\item man1\_{}MANS, man2\_{}MANS, etc., can be used to force renaming at install time. This renaming is skipped if the suffix already begins with the correct number. For instance: 

\begin{Verbatim}[frame=single]
man1_MANS = foo.man
man3_MANS = quux.3n
\end{Verbatim}

Here `foo.man' will be installed as `foo.1' but `quux.3n' will keep its name at install time. 
\end{itemize}

\item[TEXINFOS]
\ %

GNU programs traditionally use the Texinfo documentation format, not man pages. Automake has full support for Texinfo, including some additional features such as versioning and install-info support. We won't go into that here except to mention that it exists. See the Automake reference manual for more information. 
\end{description} 

Automake supports a variety of lesser-used primaries such as JAVA and LISP (and, in the next major release, PYTHON). See the reference manual for more information on these. 

\section{Programs and libraries}


The preceding primaries have all been relatively easy to use. Now we'll discuss a more complicated set, namely those used to build programs and libraries. These primaries are more complex because building a program is more complex than building a script (which often doesn't even need building at all). 


Use the PROGRAMS primary for programs, LIBRARIES for libraries, and 
LTLIBRARIES for Libtool libraries
(see chapter \ref{C_Introducing_GNU_Libtool} Introducing GNU Libtool,
page \pageref{C_Introducing_GNU_Libtool}). Here is a minimal example: 

\begin{Verbatim}[frame=single]
bin_PROGRAMS = doit
\end{Verbatim}
 This creates the program doit and arranges to install it in bindir. First make will compile `doit.c' to produce `doit.o'. Then it will link `doit.o' to create `doit'. 


Of course, if you have more than one source file, and most programs do,
then you will want to be able to list them 
somehow ({\MaQ\cH74}\z{\MgQ\cH47}\z{\McQ\cH7}\z{\MbQ\cH106}\z{\MbQ\cH31}; {\MbQ\cH224}\z{\MgQ\cH47}\z{\McQ\cH7}\z{\MbQ\cH106}\z{\MbQ\cH182}; {\MbQ\cH42}\z{\MgQ\cH47}\z{\McQ\cH7}\z{\McQ\cH105}\z{\MbQ\cH25}). You will do this via 
the program's SOURCES variable. Each program or library has a set of 
associated variables whose names are constructed by appending suffixes to 
the `normalized' name of the program. The normalized name is the name of 
the object with non-alphanumeric characters changed to 
underscores ({\MaQ\cH202}... {\MaQ\cH45}\z{\MdQ\cH149}\z{\McQ\cH40}, {\MbQ\cH35}\z{\McQ\cH124}). For 
instance, the normalized name of `quux' is `quux', but the normalized name 
of `install-info' is `install\_{}info'. Normalized names are used because 
they correspond to make syntax, and, like all macros, Automake 
propagates ({\MhQ\cH245}\z{\MgQ\cH122}, {\MaQ\cH211}\z{\MgQ\cH122}, {\MaQ\cH85}\z{\McQ\cH179}\z{\MaQ\cH102}, {\MaQ\cH102}\z{\MfQ\cH200}; {\MaQ\cH85}\z{\MgQ\cH0}\z{\MaQ\cH163}) these 
definitions into the resulting `Makefile.in'. 


So if `doit' is to be built from files `main.c' and `doit.c', we would write: 

\begin{Verbatim}[frame=single]
bin_PROGRAMS = doit
doit_SOURCES = doit.c main.c
\end{Verbatim}

The same holds for libraries. In the zlib package we might make a library called `libzlib.a'. Then we would write: 

\begin{Verbatim}[frame=single]
lib_LIBRARIES = libzlib.a
libzlib_a_SOURCES = adler32.c compress.c crc32.c deflate.c \
deflate.h gzio.c infblock.c infblock.h infcodes.c infcodes.h \
inffast.c inffast.h inffixed.h inflate.c inftrees.c \
inftrees.h infutil.c infutil.h trees.c trees.h uncompr.c \
zconf.h zlib.h zutil.c zutil.h
\end{Verbatim}

We can also do this with libtool libraries. For instance, suppose we want to build `libzlib.la' instead: 

\begin{Verbatim}[frame=single]
lib_LTLIBRARIES = libzlib.la
libzlib_la_SOURCES = adler32.c compress.c crc32.c deflate.c \
deflate.h gzio.c infblock.c infblock.h infcodes.c infcodes.h \
inffast.c inffast.h inffixed.h inflate.c inftrees.c \
inftrees.h infutil.c infutil.h trees.c trees.h uncompr.c \
zconf.h zlib.h zutil.c zutil.h
\end{Verbatim}

As you can see, making shared libraries with Automake and Libtool is just as easy as making static libraries. 


In the above example, we listed header files in the SOURCES variable. These 
are ignored (except by make 
dist\footnote{See chapter \ref{C_Rolling_Distribution_Tarballs} Rolling
Distribution Tarballs, page \pageref{C_Rolling_Distribution_Tarballs}} )
but can serve to make your `Makefile.am' a bit clearer (and 
sometimes shorter, if you aren't installing headers). 


Note that you can't use `configure' substitutions ({\MaQ\cH72}\z{\MgQ\cH23}; {\MaQ\cH72}\z{\MbQ\cH224}; {\MgQ\cH23}\z{\MfQ\cH178}) in a 
SOURCES variable. Automake needs to know the static list of files which 
can be compiled into your program. There are still various ways to 
conditionally compile files, for instance Automake conditionals or the use of 
the LDADD variable. 


The static list of files is also used in some versions of Automake's automatic dependency tracking. The general rule is that each source file which might be compiled should be listed in some SOURCES variable. If the source is conditionally compiled, it can be listed in an EXTRA variable. For instance, suppose in this example `@FOO\_{}OBJ@' is conditionally set by `configure' to `foo.o' when `foo.c' should be compiled: 


\begin{Verbatim}[frame=single]
bin_PROGRAMS = foo
foo_SOURCES = main.c
foo_LDADD = @FOO_OBJ@
foo_DEPENDENCIES = @FOO_OBJ@
EXTRA_foo_SOURCES = foo.c
\end{Verbatim}

In this case, `EXTRA\_{}foo\_{}SOURCES' is used to list sources which are 
conditionally compiled; this tells Automake that they exist
\underline{even though} ({\MdQ\cH185}\z{\MaQ\cH85}; {\McQ\cH214}\z{\MbQ\cH205}) it can't 
deduce ({\MbQ\cH198}\z{\MnQ\cH85}, {\MbQ\cH82}\z{\McQ\cH127}) their existence automatically. 


In the above example, note the use of the `foo\_{}LDADD' macro. This macro is 
used to list other object files and libraries which should be linked into 
the foo program. Each program or library has several such associated 
macros which can be used to customize the link step; here we list the 
most common ones: 

\begin{description}
\item[`\_{}DEPENDENCIES']
\ %

Extra dependencies which are added to the program's dependency list. If not 
specified, this is automatically computed based on the value of the 
program's `\_{}LDADD' macro. 

\item[`\_{}LDADD']
\ %

Extra objects which are passed to the linker. This is only used by programs
and shared libraries. 

\item[`\_{}LDFLAGS']
\ %

Flags which are passed to the linker. This is separate from `\_{}LDADD' to 
allow `\_{}DEPENDENCIES' to be auto-computed. 

\item[`\_{}LIBADD']
\ %

Like `\_{}LDADD', but used for static libraries and not programs. 
\end{description}

You aren't required to define any of these macros. 

\section{Frequently Asked Questions}


Experience has shown that there are several common questions that 
arise ({\MbQ\cH223}\z{\MbQ\cH222}, {\MaQ\cH124}\z{\MbQ\cH219}, {\MbQ\cH36}\z{\MbQ\cH65}) as people begin to use automake for their 
own projects. It seemed prudent ({\MaQ\cH246}\z{\MfQ\cH70}\z{\MbQ\cH237}, {\MaQ\cH252}\z{\MbQ\cH45}) to mention these issues here. 


Users often want to make a library (or program, but for some reason it 
\underline{comes up} ({\McQ\cH92}\z{\MbQ\cH84}\z{\MaQ\cH124}, {\McQ\cH92}\z{\McQ\cH110}\z{\McQ\cH127}) more frequently with libraries) whose 
sources live in subdirectories: 

\begin{Verbatim}[frame=single]
lib_LIBRARIES = libsub.a
libsub_a_SOURCES = subdir1/something.c ...
\end{Verbatim}

If you try this with Automake 1.4, you'll get an error: 

\begin{Verbatim}[frame=single]
$ automake
automake: Makefile.am: not supported: source file 
    subdir1/something.c is in subdirectory
\end{Verbatim}


For libraries, this problem is mostly simply solve by using libtool 
convenience libraries. For programs, there is no simple solution. Many people 
elect ({\McQ\cH178}\z{\MfQ\cH207}, {\MbQ\cH178}\z{\MaQ\cH236}) to restructure ({\MbQ\cH90}\z{\McQ\cH32}; {\McQ\cH189}\z{\MbQ\cH29}; {\McQ\cH124}\z{\MbQ\cH99}) their package in this case.

The next major release of Automake addresses ({\MaQ\cH250}\z{\MaQ\cH71}; {\MbQ\cH197}\z{\McQ\cH151}) this problem. 


Another general problem that comes up is that of setting compilation flags. Most rules have flags--for instance, compilation of C code automatically uses `CFLAGS'. However, these variables are considered user variables. Setting them in `Makefile.am' is unsafe, because the user will expect to be able to override them at will. 


To handle this, for each flag variable, Automake introduce an `AM\_{}' version which can be set in `Makefile.am'. For instance, we could set some flags for C and C++ compilation like so: 

\begin{Verbatim}[frame=single]
AM_CFLAGS = -DFOR_C
AM_CXXFLAGS = -DFOR_CXX
\end{Verbatim}



 Finally, people often ask how to compile a single source file in two different ways. For instance, the `etags.c' file which comes with Emacs can be compiled with different `-D' options to produce the etags and ctags programs. 


With Automake 1.4 this can only be done by writing your own compilation rules, like this: 

\begin{Verbatim}[frame=single]
bin_PROGRAMS = etags ctags
etags_SOURCES = etags.c
ctags_SOURCES =
ctags_LDADD = ctags.o

etags.o: etags.c
        $(CC) $(CFLAGS) -DETAGS ...

ctags.o: etags.c
        $(CC) $(CFLAGS) -DCTAGS ...
\end{Verbatim}



This is tedious ({\MdQ\cH121}\z{\McQ\cH197}\z{\MdQ\cH12}\z{\MdQ\cH226}\z{\MbQ\cH237}; {\MaQ\cH85}\z{\MaQ\cH65}\z{\MdQ\cH192}\z{\MhQ\cH4}\z{\MbQ\cH237}) and hard to maintain for 
larger programs. Automake 1.5 will support a much more natural approach: 

\begin{Verbatim}[frame=single]
bin_PROGRAMS = etags ctags
etags_SOURCES = etags.c
etags_CFLAGS = -DETAGS
ctags_SOURCES = etags.c
ctags_CFLAGS = -DCTAGS
\end{Verbatim}


\section{Multiple directories}

\underline{So far} ({\MaQ\cH131}\z{\MbQ\cH241}\z{\MaQ\cH135}\z{\MbQ\cH209}\z{\MbQ\cH162}) , we've 
only dealt (deal {\MbQ\cH237}\z{\McQ\cH172}\z{\MaQ\cH160}\z{\MbQ\cH31}\z{\McQ\cH68}\z{\McQ\cH172}\z{\MaQ\cH160}\z{\MaQ\cH125}\z{\MiQ\cH200})with single-directory 
projects. Automake can also handle projects with many directories. The 
variable `SUBDIRS' is used to list the subdirectories which should be 
built. Here is an example from Automake itself: 


\begin{Verbatim}[frame=single]
SUBDIRS = . m4 tests
\end{Verbatim}

Automake does not need to know the list of subdirectories statically, so 
there is no `EXTRA\_{}SUBDIRS' variable. You might think that Automake would 
use `SUBDIRS' to see which `Makefile.am's to scan, but it actually 
gets this information from `configure.in'. This means that, if you have a 
subdirectory which is optionally built, you should still list it 
unconditionally in your call to AC\_{}OUTPUT and then arrange for it to be 
substituted (or not, as appropriate) at configure time. 


Subdirectories are always built in the order they 
appear,\marginpar{ {\MbQ\cH29}\z{\MgQ\cH90}\z{\MjQ\cH223}\z{\MeQ\cH229}\z{\MaQ\cH89}\z{\MbQ\cH206}\z{\MbQ\cH241}\z{\MjQ\cH147}\z{\MbQ\cH237}\z{\MjQ\cH223}\z{\MeQ\cH229}, {\McQ\cH55}\z{\MbQ\cH192}\z{\McQ\cH206}\z{\McQ\cH6}\z{\MeQ\cH229}\z{\MaQ\cH134}\z{\MaQ\cH74}\z{\MaQ\cH165}\z{\MaQ\cH178}\z{\MbQ\cH237}\z{\MbQ\cH241}\z{\MjQ\cH147}\z{\MbQ\cH159}\z{\MeQ\cH229}\z{\MaQ\cH86}\z{\MaQ\cH207}\z{\McQ\cH87}{\MaQ\cH1}}
but cleaning rules (e.g., maintainer-clean) are always run in the reverse order. The reason for this odd reversal is that it is wrong to remove a file before removing all the files which depend on it. 


You can put `.' into `SUBDIRS' to control when the objects in the current directory are built, relative to the objects in the subdirectories. In the example above, targets in `.' will be built before subdirectories are built. If `.' does not appear in `SUBDIRS', it is built following all the subdirectories.

\section{Testing}


Automake also includes simple support for testing your
program.\marginpar{Automake {\MaQ\cH170}\z{\MaQ\cH74}\z{\MaQ\cH250}\z{\MbQ\cH29}\z{\MgQ\cH90}\z{\MaQ\cH124}\z{\MaQ\cH86}\z{\MbQ\cH237}\z{\McQ\cH6}\z{\MbQ\cH31}\z{\MaQ\cH100}\z{\MaQ\cH60}\z{\McQ\cH21}\z{\MaQ\cH192}\z{\MbQ\cH237}\z{\MgQ\cH191}\z{\McQ\cH116}{\MaQ\cH1}}


The most simple form of this is the `TESTS' variable. This variable holds a 
list of tests which are run when the user runs make check. Each test is 
built (if necessary) and then executed. For each test, make prints a 
single line indicating whether the test has passed or failed. Failure means 
exiting with a non-zero status, with the special exception that an exit 
status of `77'\footnote{A number chosen arbitrarily by the Automake developers.}
means that the test should be ignored. make check also prints a summary 
showing the number of passes and fails. 


Automake also supports the notion of an \textit{xfail}, which is a test which is expected to fail. Sometimes this is useful when you want to track a known failure, but you aren't prepared to fix it right away. Tests which are expected to fail should be listed in both `TESTS' and `XFAIL\_{}TESTS'. 


The special prefix `check' can be used with primaries to indicate that the objects should only be built at make check time. For example, here is how you can build a program that will only be used during the testing process: 


\begin{Verbatim}[frame=single]
check_PROGRAMS = test-program
test_program_SOURCES = ...
\end{Verbatim}

Automake also supports the use of DejaGNU, the GNU test framework. DejaGNU support can be enabled using the `dejagnu' option: 


\begin{Verbatim}[frame=single]
AUTOMAKE_OPTIONS = dejagnu
\end{Verbatim}


The resulting `Makefile.in' will include code to invoke the runtest program appropriately. 
