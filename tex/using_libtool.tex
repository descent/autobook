\chapter{Using GNU Libtool with `configure.in' and `Makefile.am'}\label{C_Using GNU Libtool with}


Although Libtool is usable by itself, either from the command line or from 
a non-make driven build system, it is also tightly integrated into Autoconf 
and Automake. This chapter discusses how to use Libtool with Autoconf and 
Automake and explains how to set up the files you write (`Makefile.am' 
and `configure.in') to take advantage of libtool. For a more in depth 
discussion of the workings of Libtool, particularly its command line interface,
See chapter \ref{C_Introducing_GNU_Libtool} Introducing GNU Libtool
(page \pageref{C_Introducing_GNU_Libtool}). Using libtool for dynamic runtime 
loading is described in See chapter \ref{C_Using_GNU_libltdl} Using GNU libltdl
(page \pageref{C_Using_GNU_libltdl}). 


Using libtool to build the libraries in a project, requires declaring your use of libtool inside the project's `configure.in' and adding the Libtool support scripts to the distribution. You will also need to amend ({\MaQ\cH94}\z{\MbQ\cH163}) the build rules in either `Makefile.am' or `Makefile.in', depending on whether you are using Automake. 

\section{Integration with `configure.in'}


Declaring your use of libtool in the project's `configure.in' is a simple matter of adding the `AC\_{}PROG\_{}LIBTOOL'(18) somewhere near the top of the file. I always put it immediately after the other `AC\_{}PROG\_{}...' macros. If you are converting an old project to use libtool, then you will also need to remove any calls to `AC\_{}PROG\_{}RANLIB'. Since Libtool will be handling all of the libraries, it will decide whether or not to call ranlib as appropriate for the build environment. 


The code generated by `AC\_{}PROG\_{}LIBTOOL' relies on the shell 
variable \$top\_{}builddir to hold the relative path to the directory which 
contains the configure script. If you are using Automake, \$top\_{}builddir 
is set in the environment by the generated `Makefile'. If you use Autoconf 
without Automake then you must ensure that \$top\_{}builddir is set before 
the call to `AC\_{}PROG\_{}LIBTOOL' in `configure.in'. 

Adding the following code to `configure.in' is often sufficient: 

\begin{verbatim}
for top_builddir in . .. ../.. $ac_auxdir $ac_auxdir/..; do
  test -f $top_builddir/configure && break
done
\end{verbatim}



Having made these changes to add libtool support to your project, you will need to regenerate the `aclocal.m4' file to pick up the macro definitions required for `AC\_{}PROG\_{}LIBTOOL', and then rebuild your configure script with these new definitions in place. After you have done that, there will be some new options available from configure: 

\begin{verbatim}
$ aclocal
$ autoconf
$ ./configure --help
...
--enable and --with options recognized:
 --enable-shared[=PKGS]  build shared libraries [yes]
 --enable-static[=PKGS]  build static libraries [yes]
 --enable-fast-install[=PKGS]  optimize for fast installation [yes]
 --with-gnu-ld           assume the C compiler uses GNU ld [no]
 --disable-libtool-lock  avoid locking (might break parallel builds)
 --with-pic              try to use only PIC/non-PIC objects [both]
\end{verbatim}

These new options allow the end user of your project some control over how they want to build the project's libraries. The opposites of each of these switches are also accepted, even though they are not listed by configure --help. You can equally pass, `--disable-fast-install' or `--without-gnu-ld' for example. 

\subsection{Extra Configure Options}
What follows is a list that describes the more commonly used options that are 
automatically added to configure, by virtue of using `AC\_{}PROG\_{}LIBTOOL'
in your `configure.in'. The Libtool Manual distributed with Libtool releases 
always contains the most up to date information about libtool options: 

\begin{description}
\item[`--enable-shared']
\item[`--enable-static']
\ 
%

More often invoked as `--disable-shared' or equivalently `--enable-shared=no' these switches determine whether libtool should build shared and/or static libraries in this package. If the installer is short of disk space, they might like to build entirely without static archives. To do this they would use:

\begin{verbatim}
$ ./configure --disable-static
\end{verbatim}

Sometimes it is desirable to configure several related packages with the same command line. From a scheduled build script or where subpackages with their own configure scripts are present, for example. The `--enable-shared' and `--enable-static' switches also accept a list of package names, causing the option to be applied to packages whose name is listed, and the opposite to be applied to those not listed. 


By specifying: 

\begin{verbatim}
$ ./configure --enable-static=libsnprintfv,autoopts
\end{verbatim}


libtool would pass `--enable-static' to only the packages named libsnprintfv 
and autoopts in the current tree. Any other packages configured would 
effectively be passed `--disable-static'. Note that this doesn't 
necessarily mean that the packages must honour these options. Enabling static 
libraries for a package which consists of only dynamic modules makes no sense,
and the package author would probably have decided to ignore such requests,
See section 11.1.2 Extra Macros for Libtool. 

\item[`--enable-fast-install']
\ 

%
On some machines, libtool has to relink executables when they are installed, See section 10.7 Installing an Executable. Normally, when an end user builds your package, they will probably type:

\begin{Verbatim}[frame=single]
$ ./configure
$ make
$ make install
\end{Verbatim}



libtool will build executables suitable for copying into their respective installation destinations, obviating the need for relinking them on those hosts which would have required it. Whenever libtool links an executable which uses shared libraries, it also creates a wrapper script which ensures that the environment is correct for loading the correct libraries, See section 10.5 Executing Uninstalled Binaries. On those hosts which require it, the wrapper script will also relink the executable in the build tree if you attempt to run it from there before installation. 


Sometimes this behaviour is not what you want, particularly if you are developing the package and not installing between test compilations. By passing `--disable-fast-install', the default behaviour is reversed; executables will be built so that they can be run from the build tree without relinking, but during installation they may be relinked. 


You can pass a list of executables as the argument to `--enable-fast-install' to determine which set of executables will not be relinked at installation time (on the hosts that require it). By specifying: 

\begin{verbatim}
$ ./configure --enable-fast-install=autogen
\end{verbatim}



 The autogen executable will be linked for fast installation (without being relinked), and any other executables in the build tree will be linked for fast execution from their build location. This is useful if the remaining executables are for testing only, and will never be installed. 


Most machines do not require that executables be relinked in this way, and in these cases libtool will link each executable once only, no matter whether `--disable-fast-install' is used. 

\item[`--with-gnu-ld']
\ 

%
This option is used to inform libtool that the C compiler is using GNU ld as its linker. It is more often used in the opposite sense when both gcc and GNU ld are installed, but gcc was built to use the native linker. libtool will probe the system for GNU ld, and assume that it is used by gcc if found, unless `--without-gnu-ld' is passed to configure. 


\item[`--disable-libtool-lock']
\ 

%
In normal operation, libtool will build two objects for every source file in a package, one PIC(19) and one non-PIC. With gcc and some other compilers, libtool can specify a different output location for the PIC object: 

\begin{Verbatim}[frame=single]
$ libtool gcc -c shell.c
gcc -c -pic -DPIC shell.c -o .libs/shell.lo
gcc -c foo.c -o shell.o >/dev/null 2>&1
\end{Verbatim}



 When using a compiler that doesn't accept both `-o' and `-c' in the same command, libtool must compile first the PIC and then the non-PIC object to the same destination file and then move the PIC object before compiling the non-PIC object. This would be a problem for parallel builds, since one file might overwrite the other. libtool uses a simple shell locking mechanism to avoid this eventuality. 


If you find yourself building in an environment that has such a compiler, and not using parallel make, then the locking mechanism can be safely turned off by using `--disable-libtool-lock' to gain a little extra speed in the overall compilation. 

\item[`--with-pic']
\ 

%
In normal operation, Libtool will build shared libraries from PIC objects and static archives from non-PIC objects, except where one or the other is not provided by the target host. By specifying `--with-pic' you are asking libtool to build static archives from PIC objects, and similarly by specifying `--without-pic' you are asking libtool to build shared libraries from non-PIC objects. 

libtool will only honour this flag where it will produce a working library, otherwise it reverts to the default. 
\end{description}

\subsection{Extra Macros for Libtool}
There are several macros which can be added to `configure.in' which will 
change the default behaviour of libtool. If they are used they must appear 
before the call to the `AC\_{}PROG\_{}LIBTOOL' macro. Note that these macros 
only change the default behaviour, and options passed in to configure on the 
command line will always override the defaults. The most up to date 
information about these macros is available from the Libtool Manual. 

\begin{description}
\item[`AC\_{}DISABLE\_{}FAST\_{}INSTALL']
\ 

%
This macro tells libtool that on platforms which require relinking at install time, it should build executables so that they can be run from the build tree at the expense of relinking during installation, as if `--disable-fast-install' had been passed on the command line. 


\item[`AC\_{}DISABLE\_{}SHARED']
\item[`AC\_{}DISABLE\_{}STATIC']
\ 

%
These macros tell libtool to not try and build either shared or static libraries respectively. libtool will always try to build something however, so even if you turn off static library building in `configure.in', building your package for a target host without shared library support will fallback to building static archives.
\end{description}

The time spent waiting for builds during development can be reduced a little by including these macros temporarily. Don't forget to remove them before you release the project though! 


In addition to the macros provided with `AC\_{}PROG\_{}LIBTOOL', there are a 
few shell variables that you may need to set yourself, depending on the 
structure of your project: 

\begin{description}
\item[`LTLIBOBJS']
\ 

%
If your project uses the `AC\_{}REPLACE\_{}FUNCS' macro, or any of the 
other macros which add object names to the `LIBOBJS' variable, you will also 
need to provide an equivalent `LTLIBOBJS' definition. At the moment, you 
must do it manually, but needing to do that is considered to be a bug and 
will fixed in a future release of Autoconf. The manual generation 
of `LTLIBOBJS' is a simple matter of replacing the names of the objects 
mentioned in `LIBOBJS' with equivalent .lo suffixed Libtool object names.
The easiest way to do this is to add the following snippet to 
your `configure.in' near the end, just before the call to `AC\_{}OUTPUT'. 
 
\begin{verbatim}
Xsed="sed -e s/^X//"
LTLIBOBJS=`echo X"$LIBOBJS"|\
           [$Xsed -e "s,\.[^.]* ,.lo ,g;s,\.[^.]*$,.lo,"]`
AC_SUBST(LTLIBOBJS)
\end{verbatim}

The Xsed is not usually necessary, though it can prevent problems with 
the echo command in the event that one of the `LIBOBJS' files begins with 
a `-' character. It is also a good habit to write shell code like this, as 
it will avoid problems in your programs. 

\item[`LTALLOCA']
\ 

%
If your project uses the `AC\_{}FUNC\_{}ALLOCA' macro, you will need to provide a definition of `LTALLOCA' equivalent to the `ALLOCA' value provided by the macro. 


 
\begin{verbatim}
Xsed="sed -e s/^X//"
LTALLOCA=`echo X"$ALLOCA"|[$Xsed -e "s,\.$[^.]*,.lo,g"]`
AC_SUBST(LTALLOCA)
\end{verbatim}



Obviously you don't need to redefine Xsed if you already use 
it for `LTLIBOBJS' above. 



\item[`LIBTOOL\_{}DEPS']
\ 

%
To help you write make rules for automatic updating of the Libtool 
configuration files, you can use the value of `LIBTOOL\_{}DEPS' after 
the call to `AC\_{}PROG\_{}LIBTOOL': 

\begin{Verbatim}[frame=single]
AC_PROG_LIBTOOL
AC_SUBST(LIBTOOL_DEPS)
\end{Verbatim}



 Then add the following to the top level `Makefile.in': 



 

\begin{Verbatim}[frame=single]
libtool: @LIBTOOL_DEPS@
        cd $(srcdir) && \
          $(SHELL) ./config.status --recheck
\end{Verbatim}



 If you are using automake in your project, it will generate equivalent rules automatically. You don't need to use this except in circumstances where you want to use libtool and autoconf, but not automake.
\end{description}

\section{Integration with `Makefile.am'}


Automake supports Libtool libraries in two ways. It can help you to build the Libtool libraries themselves, and also to build executables which link against Libtool libraries. 

\subsection{Creating Libtool Libraries with Automake}\label{SS_Creating_Libtool_Libraries_with_Automake}


Continuing in the spirit of making Libtool library management look like native static archive management, converting a `Makefile.am' from static archive use to Libtool library use is a matter of changing the name of the library, and adding a Libtool prefix somewhere. For example, a `Makefile.am' for building a static archive might be: 

\begin{Verbatim}[frame=single]
lib_LIBRARIES      = libshell.a
libshell_a_SOURCES = object.c subr.c symbol.c
\end{Verbatim}



 This would build a static archive called `libshell.a' consisting of the objects `object.o', `subr.o' and `bar.o'. To build an equivalent Libtool library from the same objects, you change this to: 



\begin{Verbatim}[frame=single]
lib_LTLIBRARIES     = libshell.la
libshell_la_SOURCES = object.c subr.c symbol.c
\end{Verbatim}

The only changes are that the library is now named with a .la suffix, and the Automake primary is now `LTLIBRARIES'. Note that since the name of the library has changed, you also need to use `libshell\_{}la\_{}SOURCES', and similarly for any other Automake macros which used to refer to the old archive. As for native libraries, Libtool library names should begin with the letters `lib', so that the linker will be able to find them when passed `-l' options. 


Often you will need to add extra objects to the library as determined by configure, but this is also a mechanical process. When building native libraries, the `Makefile.am' would have contained:

\begin{Verbatim}[frame=single]
libshell_a_LDADD = xmalloc.o @LIBOBJS@
\end{Verbatim}

 To add the same objects to an equivalent Libtool library would require: 
 

\begin{Verbatim}[frame=single]
libshell_la_LDADD = xmalloc.lo @LTLIBOBJS@
\end{Verbatim}



That is, objects added to a Libtool library must be Libtool objects
(with a .lo) suffix. You should add code to `configure.in' to ensure 
that `LTALLOCA' and `LTLIBOBJS' are set appropriately,
See section 11.1.2 Extra Macros for Libtool. Automake will take care of 
generating appropriate rules for building the Libtool objects mentioned in 
an `LDADD' macro. 


If you want to pass any additional flags to libtool when it is building, you use the `LDFLAGS' macro for that library, like this: 



 

\begin{Verbatim}[frame=single]
libshell_la_LDFLAGS = -version-info 1:0:1
\end{Verbatim}



 For a detailed list of all the available options, see section `Link mode' in The Libtool Manual. 

Libtool's use of `-rpath' has been a point of contention for some users, since it prevents you from moving shared libraries to another location in the library search path. Or, at least, if you do, all of the executables that were linked with `-rpath' set to the old location will need to be relinked. 

We (the Libtool maintainers) assert that always using `-rpath' is a good thing: Mainly because you can guarantee that any executable linked with `-rpath' will find the correct version of the library, in the rpath directory, that was intended when the executable was linked. Library versions can still be managed correctly, and will be found by the run time loader, by installing newer versions to the same directory. Additionally, it is much harder for a malicious user to leave a modified copy of system library in a directory that someone might wish to list in their `LD\_{}LIBRARY\_{}PATH' in the hope that some code they have written will be executed unexpectedly. 


The argument against `-rpath' was instigated when one of the GNU/Linux distributions moved some important system libraries to another directory to make room for a different version, and discovered that all of the executables that relied on these libraries and were linked with Libtool no longer worked. Doing this was, arguably, bad system management -- the new libraries should have been placed in a new directory, and the old libraries left alone. Refusing to use `-rpath' incase you want to restructure the system library directories is a very weak argument. 


The `-rpath' option (which is required for Libtool libraries) is automatically supplied by automake based on the installation directory specified with the library primary.  


\begin{Verbatim}[frame=single]
lib_LTLIBRARIES = libshell.la
\end{Verbatim}

The example would use the value of the make macro \$(libdir) as the argument 
to `-rpath', since that is where the library will be installed. 


A few of the other options you can use in the library `LDFLAGS' are: 

\begin{description}
\item[`-no-undefined']
\ 

%
Modern architectures allow us to create shared libraries with undefined symbols, provided those symbols are resolved (usually by the executable which loads the library) at runtime. Unfortunately, there are some architectures (notably AIX and Windows) which require that all symbols are resolved when the library is linked. If you know that your library has no unresolved symbols at link time, then adding this option tells libtool that it will be able to build a shared library, even on architectures which have this requirement. 

\item[`-static']
\ 

%
Using this option will force libtool to build only a static archive for this 
library. 

\item[`-release']
\ 

%
On occasion, it is desirable to encode the release number of a library 
into its name. By specifying the release number with this option, libtool 
will build a library that does this, but will break binary compatibility 
for each change of the release number. By breaking binary compatibility this 
way, you negate the possibility of fixing bugs in installed programs by 
installing an updated shared library. You should probably be 
using `-version-info' instead.

\begin{Verbatim}[frame=single]
libshell_la_LDFLAGS = -release 27
\end{Verbatim}

The above fragment might create a library called `libshell-27.so.0.0.0' for example. 


\item[`-version-info']
\ 

%
Set the version number of the library according to the native versioning rules 
based on the numbers supplied, See section \ref{S_Library_Versioning} Library
Versioning (page \pageref{S_Library_Versioning}). You need to be aware that 
the library version number is for the use of the runtime loader, and is 
completely unrelated to the release number of your project. If you really want 
to encode the project release into the library, you can use `-release' to 
do it.Set the version number of the library according to the native versioning 
rules based on the numbers supplied, See section \ref{S_Library_Versioning}
Library Versioning, (page \pageref{S_Library_Versioning}. You need to be aware 
that the library version number is for the use of the runtime loader, and is 
completely unrelated to the release number of your project. If you really 
want to encode the project release into the library, you can use `-release' 
to do it.

\end{description}

If this option is not supplied explicitly, it defaults to `-version-info 0:0:0'. 


Historically, the default behaviour of Libtool was as if `-no-undefined' was always passed on the command line, but it proved to be annoying to developers who had to constantly turn it off so that their ELF libraries could be featureful. Now it has to be defined explicitly if you need it. 

There are is a tradeoff: 

\begin{itemize}

\item If you don't specify `-no-undefined', then Libtool will not build shared libraries on platforms which don't allow undefined symbols at link time for such a library. 

\item It is only safe to specify this flag when you know for certain that all of the libraries symbols are defined at link time, otherwise the `-no-undefined' link will appear to work until it is tried on a platform which requires all symbols to be defined. Libtool will try to link the shared library in this case (because you told it that you have not left any undefined symbols), but the link will fail, because there are undefined symbols in spite of what you told Libtool. 
\end{itemize}

For more information about this topic, see 18.3 Portable Library Design. 

\subsection{Linking against Libtool Libraries with Automake}


Once you have set up your `Makefile.am' to create some Libtool libraries. you will want to link an executable against them. You can do this easily with automake by using the program's qualified `LDADD' macro: 

\begin{Verbatim}[frame=single]
bin_PROGRAMS  = shell
shell_SOURCES = shell.c token.l
shell_LDADD   = libshell.la
\end{Verbatim}



 This will choose either the static or shared archive from the `libshell.la' Libtool library depending on the target host and any Libtool mode switches metioned in the `Makefile.am', or passed to configure. The chosen archive will be linked with any objects generated from the listed sources to make an executable. Note that the executable itself is a hidden file, and that in its place libtool creates a wrapper script, See section 10.5 Executing Uninstalled Binaries. 


As with the Libtool libraries, you can pass additional switches for the libtool invocation in the qualified `LDFLAGS' macros to control how the shell executable is linked: 

\begin{description}
\item[`-all-static']
\ 

%
Always choose static libraries where possible, and try to create a completely statically linked executable.

\item[`-no-fast-install'] 
\ 

%
If you really want to use this flag on some targets, you can pass it in an `LDFLAGS' macro. This is not overridden by the configure `--enable-fast-install' switch. Executables built with this flag will not need relinking to be executed from the build tree on platforms which might have otherwise required it. 

\item[`-no-install'] 
\ 

%
You should use this option for any executables which are used only for testing, or for generating other files and are consequently never installed. By specifying this option, you are telling Libtool that the executable it links will only ever be executed from where it is built in the build tree. Libtool is usually able to considerably speed up the link process for such executables. 

\item[`-static'] 
\ 

%
This switch is similar to `-all-static', except that it applies to only the uninstalled Libtool libraries in the build tree. Where possible the static archive from these libraries is used, but the default linking mode is used for libraries which are already installed. 
\end{description}

When debugging an executable, for example, it can be useful to temporarily use: 

\begin{Verbatim}[frame=single]
shell_LDFLAGS = -all-static
\end{Verbatim}


You can pass Libtool link options to all of the targets in a given 
directory by using the unadorned `LDFLAGS' macro: 

\begin{Verbatim}[frame=single]
LDFLAGS = -static
\end{Verbatim}

 This is best reserved for directories which have targets of the same type, all Libtool libraries or all executables for instance. The technique still works in a mixed target type directory, and libtool will ignore switches which don't make sense for particular targets. It is less maintainable, and makes it harder to understand what is going on if you do that though. 

\section{Using libtoolize}


Having made the necessary editions in `configure.in' and `Makefile.am', all that remains is to add the Libtool infrastructure to your project. 


First of all you must ensure that the correct definitions for the new macros you use in `configure.in' are added to `aclocal.m4', See section C. Generated File Dependencies. At the moment, the safest way to do this is to copy `libtool.m4' from the installed libtool to `acinclude.m4' in the toplevel source directory of your package. This is to ensure that when your package ships, there will be no mismatch errors between the M4 macros you provided in the version of libtool you built the distribution with, versus the version of the Libtool installation in another developer's environment. In a future release, libtool will check that the macros in aclocal.m4 are from the same Libtool distribution as the generated libtool script. 


\begin{verbatim}
$ cp /usr/share/libtool/libtool.m4 ./acinclude.m4
$ aclocal
\end{verbatim} 

By naming the file `acinclude.m4' you ensure that aclocal can see it and will use macros from it, and that automake will add it to the distribution when you create the tarball. 


Next, you should run libtoolize, which adds some files to your distribution that are required by the macros from `libtool.m4'. In particular, you will get `ltconfig'(20) and `ltmain.sh' which are used to create a custom libtool script on the installer's machine. 


If you do not yet have them, libtoolize will also add `config.guess' and `config.sub' to your distribution. Sometimes you don't need to run libtoolize manually, since automake will run it for you when it sees the changes you have made to `configure.in', as follows: 

\begin{verbatim}
$ automake --add-missing
automake: configure.in: installing ./install-sh
automake: configure.in: installing ./mkinstalldirs
automake: configure.in: installing ./missing
configure.in: 8: required file ./ltconfig not found
\end{verbatim}



 The error message in the last line is an abberation. If it was consistant with the other lines, it would say: 



 
\begin{verbatim}
automake: configure.in: installing ./ltconfig
automake: configure.in: installing ./ltmain.sh
automake: configure.in: installing ./config.guess
automake: configure.in: installing ./config.sub
\end{verbatim}

 But the effect is the same, and the files are correctly added to the distribution despite the misleading message. 


Before you release a distribution of your project, it is wise to get the 
latest versions of `config.guess' and `config.sub' from the GNU site(21),
since they may be newer than the versions automatically added by libtoolize 
and automake. Note that automake --add-missing will give you its own 
version of these two files if `AC\_{}PROG\_{}LIBTOOL' is not used in the 
project `configure.in', but will give you the versions shipped with libtool 
if that macro is present! 

\section{Library Versioning}\label{S_Library_Versioning}


It is important to note from the outset that the version number of your project is a very different thing to the version number of any libraries shipped with your project. It is a common error for maintainers to try to force their libraries to have the same version number as the current release version of the package as a whole. At best, they will break binary compatibility unnecessarily, so that their users won't gain the benefits of the changes in their latest revision without relinking all applications that use it. At worst, they will allow the runtime linker to load binary incompatible libraries, causing applications to crash. 


Far better, the Libtool versioning system will build native shared libraries with the correct native library version numbers. Although different architectures use various numbering schemes, Libtool abstracts these away behind the system described here. The various native library version numbering schemes are designed so that when an executable is started, the runtime loader can, where appropriate, choose a more recent installed library version than the one with which the executable was actually built. This allows you to fix bugs in your library, and having built it with the correct Libtool version number, have those fixes propogate into any executables that were built with the old buggy version. This can only work if the runtime loader can tell whether it can load the new library into the old executable and expect them to work together. The library version numbers give this information to the runtime loader, so it is very important to set them correctly. 


The version scheme used by Libtool tracks interfaces, where an interface is the set of exported entry points into the library. All Libtool libraries start with `-version-info' set to `0:0:0' -- this will be the default version number if you don't explicitly set it on the Libtool link command line. The meaning of these numbers (from left to right) is as follows: 

\begin{description}
\item[current]
\ 

%
The number of the current interface exported by the library. A current value of `0', means that you are calling the interface exported by this library interface 0. 

\item[revision]
\ 

%
The implementation number of the most recent interface exported by this library. In this case, a revision value of `0' means that this is the first implementation of the interface. 

If the next release of this library exports the same interface, but has a different implementation (perhaps some bugs have been fixed), the revision number will be higher, but current number will be the same. In that case, when given a choice, the library with the highest revision will always be used by the runtime loader. 


\item[age]
\ 

%
The number of previous additional interfaces supported by this library. If age were `2', then this library can be linked into executables which were built with a release of this library that exported the current interface number, current, or any of the previous two interfaces. 
\end{description}

By definition age must be less than or equal to current. At the outset, only the first ever interface is implemented, so age can only be `0'. 


For later releases of a library, the `-version-info' argument needs to be set correctly depending on any interface changes you have made. This is quite straightforward when you understand what the three numbers mean: 

\begin{enumerate}
\item If you have changed any of the sources for this library, the revision number must be incremented. This is a new revision of the current interface. 

\item If the interface has changed, then current must be incremented, and revision reset to `0'. This is the first revision of a new interface. 

\item If the new interface is a superset of the previous interface (that is, if the previous interface has not been broken by the changes in this new release), then age must be incremented. This release is backwards compatible with the previous release. 

\item If the new interface has removed elements with respect to the previous interface, then you have broken backward compatibility and age must be reset to `0'. This release has a new, but backwards incompatible interface. 
\end{enumerate}

For example, if the next release of the library included some new commands for an existing socket protocol, you would use -version-info 1:0:1. This is the first revision of a new interface. This release is backwards compatible with the previous release. 

Later, you implement a faster way of handling part of the algorithm at the core of the library, and release it with -version-info 1:1:1. This is a new revision of the current interface. 


Unfortunately the speed of your new implementation can only be fully exploited by changing the API to access the structures at a lower level, which breaks compatibility with the previous interface, so you release it as -version-info 2:0:0. This release has a new, but backwards incompatible interface. 

When deciding which numbers to change in the -version-info argument for a new release, you must remember that an interface change is not limited to the API of the library. The notion of an interface must include any method by which a user (code or human) can interact with the library: adding new builtin commands to a shell library; the format used in an output file; the handshake protocol required for a client connecting over a socket, and so on. 


Additionally, If you use a development model which has both a stable and an unstable tree being developed in parallel, for example, and you don't mind forcing your users to relink all of the applications which use one of your Libtool libraries every time you make a release, then libtool provides the `-release' flag to encode the project version number in the name of the library, See section 11.2.1 Creating Libtool Libraries with Automake. This can save you library compatibility problems later if you need to, say, make a patch release of an older revision of your library, but the library version number that you should use has already been taken by another earlier release. In this case, you could be fairly certain that library releases from the unstable branch will not be binary compatible with the stable releases, so you could make all the stable releases with `-release 1.0' and begin the first unstable release with `-release 1.1'. 

\section{Convenience Libraries}


Sometimes it is useful to group objects together in an intermediate stage of a project's compilation to provide a useful handle for that group without having to specify all of the individual objects every time. Convenience libraries are a portable way of creating such a partially linked object: Libtool will handle all of the low level details in a way appropriate to the target host. This section describes the use of convenience libraries in conjunction with Automake. The principles of convenience libraries are discussed in Creating Convenience Libraries. 


The key to creating Libtool convenience libraries with Automake is to use the
`noinst\_{}LTLIBRARIES' macro. For the Libtool libraries named in this macro, Automake will create Libtool convenience libraries which can subsequently be linked into other Libtool libraries. 


In this section I will create two convenience libraries, each in their own subdirectory, and link them into a third Libtool library, which is ultimately linked into an application. 


If you want to follow this example, you should create a directory structure to hold the sources by running the following shell commands:

\begin{verbatim}
$ mkdir convenience
$ cd convenience
$ mkdir lib
$ mkdir replace
\end{verbatim}



The first convenience library is built from two source files in the `lib' 
subdirectory.

\begin{enumerate}
\item `source.c': 

\begin{Verbatim}[frame=single]
#if HAVE_CONFIG_H
#  include <config.h>
#endif

#if HAVE_MATH_H
#  include <math.h>
#endif

void foo (double argument)
{
 printf ("cos (%g) => %g\n", argument, cos (argument));
}
\end{Verbatim}

This file defines a single function to display the cosine of its argument on standard output, and consequently relies on an implementation of the cos function from the system libraries. Note the conditional inclusion of `config.h', which will contain a definition of `HAVE\_{}MATH\_{}H' if `configure' discovers a `math.h' system header (the usual location for the declaration of cos). The `HAVE\_{}CONFIG\_{}H' guard is by convention, so that the source can be linked by passing the preprocessor macro definitions to the compiler on the command line -- if `configure.in' does not use `AM\_{}CONFIG\_{}HEADER' for instance. 


\item `source.h':

\begin{Verbatim}[frame=single]
extern void foo        (double argument);
\end{Verbatim}

For brevity, there is no \#ifndef SOURCE\_{}H guard. The header is not installed,
so you have full control over where it is \#includeed, and in any case,
function declarations can be safely repeated if the header is accidentally 
processed more than once. In a real program, it would be better to list the 
function parameters in the declaration so that the compiler can do type 
checking. This would limit the code to working only with ANSI compilers,
unless you also use a PARAMS macro to conditionally preprocess away the 
parameters when a K\&R compiler is used. These details are beyond the scope 
of this convenience library example, but are described in full in 9.1.6 K\&R 
Compilers. 
\end{enumerate}

You also need a `Makefile.am' to hold the details of how this convenience library is linked: 



 

\begin{Verbatim}[frame=single]
## Process this file with automake to produce Makefile.in

noinst_LTLIBRARIES        = library.la
library_la_SOURCES        = source.c source.h
library_la_LIBADD        = -lm
\end{Verbatim}




The `noinst\_{}LTLIBRARIES' macro names the Libtool convenience libraries to be built in this directory, `library.la'. Although not required for compilation, `source.h' is listed in the `SOURCES' macro of `library.la' so that correct source dependencies are generated, and so that it is added to the distribution tarball by automake's `dist' rule. 


Finally, since the foo function relies on the cos function from the system math library, `-lm' is named as a required library in the `LIBADD' macro. As with all Libtool libraries, interlibrary dependencies are maintained for convenience libraries so that you need only list the libraries you are using directly when you link your application later. The libraries used by those libraries are added by Libtool. 


The parent directory holds the sources for the main executable, `main.c',
and for a (non-convenience) Libtool library, `error.c' \& `error.h'. 


Like `source.h', the functions exported from the Libtool library `liberror.la' are listed in `error.h': 

\begin{Verbatim}[frame=single]
extern void gratuitous          (void);
extern void set_program_name    (char *path);
extern void error               (char *message);
\end{Verbatim}





 The corresponding functon definitions are in `error.c': 

\begin{Verbatim}[frame=single]
#include <stdio.h>

#include "source.h"

static char *program_name = NULL;

void gratuitous (void)
{
 /* Gratuitous display of convenience library functionality! */
 double argument = 0.0;
 foo (argument);
}

void set_program_name (char *path)
{
 if (!program_name)
  program_name = basename (path);
}

void error (char *message)
{
 fprintf (stderr, "%s: ERROR: %s\n", program_name, message);
 exit (1);
}
\end{Verbatim}

The gratuitous() function calls the foo() function defined in the `library.la' convenience library in the `lib' directory, hence `source.h' is included. 


The definition of error() displays an error message to standard error, along with the name of the program, program\_{}name, which is set by calling set\_{}program\_{}name(). This function, in turn, extracts the basename of the program from the full path using the system function, basename(), and stores it in the library private variable, program\_{}name. 


Usually, basename() is part of the system C library, though older systems did not include it. Because of this, there is no portable header file that can be included to get a declaration, and you might see a harmless compiler warning due to the use of the function without a declaration. The alternative would be to add your own declaration in `error.c'. The problem with this approach is that different vendors will provide slightly different declarations (with or without const for instance), so compilation will fail on those architectures which do provide a declaration in the system headers that is different from the declaration you have guessed. 


For the benefit of architectures which do not have an implementation of the basename() function, a fallback implementation is provided in the `replace' subdirectory. The file `basename.c' follows: 

\begin{Verbatim}[frame=single]
#if HAVE_CONFIG_H
#  include <config.h>
#endif

#if HAVE_STRING_H
#  include <string.h>
#elif HAVE_STRINGS_H
#  include <strings.h>
#endif

#if !HAVE_STRRCHR
#  ifndef strrchr
#    define strrchr rindex
#  endif
#endif

char* basename (char *path)
{
 /* Search for the last directory separator in PATH. */
 char *basename = strrchr (path, '/');
  
 /* If found, return the address of the following character,
    or the start of the parameter passed in.  */
 return basename ? ++basename : path;
}

\end{Verbatim}

For brevity, the implementation does not use any const declarations which 
would be good style for a real project, but would need to be checked at 
configure time in case the end user needs to compile the package with a K\&R
compiler. 


The use of strrchr() is noteworthy. Sometimes it is declared in `string.h', otherwise it might be declared in `strings.h'. BSD based Unices, on the other hand, do not have this function at all, but provide an equivalent function, rindex(). The preprocessor code at the start of the file is designed to cope with all of these eventualities. The last block of preprocessor code assumes that if strrchr is already defined that it holds a working macro, and does not redefine it. 


`Makefile.am' contains: 

\begin{Verbatim}[frame=single]
## Process this file with automake to produce Makefile.in

noinst_LTLIBRARIES      = libreplace.la
libreplace_la_SOURCES   = 
libreplace_la_LIBADD    = @LTLIBOBJS@
\end{Verbatim}

Once again, the `noinst\_{}LTLIBRARIES' macro names the convenience library,
 `libreplace.la'. By default there are no sources, since we expect to have a system definition of basename(). Additional Libtool objects which should be added to the library based on tests at configure time are handled by the `LIBADD' macro. `LTLIBOBJS' will contain `basename.lo' if the system does not provide basename, and will be empty otherwise. Illustrating another feature of convenience libraries: on many architectures, `libreplace.la' will contain no objects. 


Back in the toplevel project directory, all of the preceding objects are 
combined by another `Makefile.am': 

\begin{Verbatim}[frame=single]
## Process this file with automake to produce Makefile.in

AUTOMAKE_OPTIONS        = foreign

SUBDIRS                 = replace lib .

CPPFLAGS                = -I$(top_srcdir)/lib

include_HEADERS         = error.h

lib_LTLIBRARIES         = liberror.la
liberror_la_SOURCES     = error.c
liberror_la_LDFLAGS     = -no-undefined -version-info 0:0:0
liberror_la_LIBADD      = replace/libreplace.la lib/library.la

bin_PROGRAMS            = convenience
convenience_SOURCES     = main.c
convenience_LDADD       = liberror.la
\end{Verbatim}

The initial `SUBDIRS' macro is necessary to ensure that the libraries in the subdirectories are built before the final library and executable in this directory. 


Notice that I have not listed `error.h' in `liberror\_{}la\_{}SOURCES' this time, since `liberror.la' is an installed library, and `error.h' defines the public interface to that library. Since the `liberror.la' Libtool library is installed, I have used the `-version-info' option, and I have also used `-no-undefined' so that the project will compile on architectures which require all library symbols to be defined at link time -- the reason program\_{}name is maintained in `liberror' rather than `main.c' is so that the library does not have a runtime dependency on the executable which links it. 


The key to this example is that by linking the `libreplace.la' and `library.la' convenience libraries into `liberror.la', all of the objects in both convenience libraries are compiled into the single installed library, `liberror.la'. Additionally, all of the inter-library dependencies of the convenience libraries (`-lm', from `library.la') are propogated to `liberror.la'. 


A common difficulty people experience with Automake is knowing when to use a `LIBADD' primary versus a `LDADD' primary. A useful mnemonic is: `LIBADD' is for ADDitional LIBrary objects. `LDADD' is for ADDitional linker (LD) objects. 

The executable, `convenience', is built from `main.c', and requires only `liberror.la'. All of the other implicit dependencies are encoded within `liberror.la'. Here is `main.c': 

\begin{Verbatim}[frame=single]
#include <stdio.h>
#include "error.h"

int
main (int argc, char *argv[])
{
  set_program_name (argv[0]);
  gratuitous ();
  error ("This program does nothing!");
}
\end{Verbatim}

The only file that remains before you can compile the example is `configure.in': 

\begin{Verbatim}[frame=single]
# Process this file with autoconf to create configure.

AC_INIT(error.c)
AM_CONFIG_HEADER(config.h)
AM_INIT_AUTOMAKE(convenience, 1.0)

AC_PROG_CC
AM_PROG_LIBTOOL

AC_CHECK_HEADERS(math.h)
AC_CHECK_HEADERS(string.h strings.h, break)

AC_CHECK_FUNCS(strrchr)
AC_REPLACE_FUNCS(basename)

Xsed="sed -e s/^X//"
LTLIBOBJS=echo X"$LIBOBJS" | \
    $Xsed -e "s,\.[^.]* ,.lo ,g;s,\.[^.]*\$,.lo,"`
AC_SUBST(LTLIBOBJS)

AC_OUTPUT(replace/Makefile lib/Makefile Makefile)
\end{Verbatim}

There are checks for all of the features used by the sources in the project: `math.h' and either `string.h' or `strings.h'; the existence of strrchr (after the tests for string headers); adding `basename.o' to `LIBOBJS' if there is no system implementation; and the shell code to set `LTLIBOBJS'. 


With all the files in place, you can now bootstrap the project:

\begin{Verbatim}
$ ls -R
.:
Makefile.am  configure.in  error.c  error.h  lib  main.c  replace

lib:
Makefile.am  source.c  source.h

replace:
Makefile.am  basename.c
$ aclocal
$ autoheader
$ automake --add-missing --copy
automake: configure.in: installing ./install-sh
automake: configure.in: installing ./mkinstalldirs
automake: configure.in: installing ./missing
configure.in: 7: required file ./ltconfig not found
$ autoconf
$ ls -R
.:
Makefile.am   config.h.in   error.c     ltconfig   mkinstalldirs
Makefile.in   config.sub    error.h     ltmain.sh  replace
aclocal.m4    configure     install-sh  main.c     
config.guess  configure.in  lib         missing

lib:
Makefile.am  Makefile.in  source.c  source.h

replace:
Makefile.am  Makefile.in  basename.c
\end{Verbatim}

With these files in place, the package can now be configured:

\begin{verbatim}
$ ./configure
...
checking how to run the C preprocessor... gcc -E
checking for math.h... yes
checking for string.h... yes
checking for strrchr... yes
checking for basename... yes
updating cache ./config.cache
creating ./config.status
creating replace/Makefile
creating lib/Makefile
creating Makefile
creating config.h
\end{verbatim}

Notice that my host has an implementation of basename(). 


Here are the highlights of the compilation itself:

\begin{verbatim}
$ make
Making all in replace
make[1]: Entering directory /tmp/replace
/bin/sh ../libtool --mode=link gcc  -g -O2  -o libreplace.la     
rm -fr .libs/libreplace.la .libs/libreplace.* .libs/libreplace.*
ar cru .libs/libreplace.al
ranlib .libs/libreplace.al
creating libreplace.la
(cd .libs && rm -f libreplace.la && ln -s ../libreplace.la \
libreplace.la)
make[1]: Leaving directory /tmp/replace
\end{verbatim}

Here the build descends into the `replace' subdirectory and 
creates `libreplace.la', which is empty on my host since I don't need an 
implementation of basename(): 

\begin{verbatim}
Making all in lib
make[1]: Entering directory /tmp/lib
/bin/sh ../libtool --mode=compile gcc -DHAVE_CONFIG_H  -I. -I. \
-g -O2 -c source.c
rm -f .libs/source.lo
gcc -DHAVE_CONFIG_H -I. -I. -g -O2 -c -fPIC -DPIC source.c \
-o .libs/source.lo
gcc -DHAVE_CONFIG_H -I. -I. -g -O2 -c source.c \
-o source.o >/dev/null 2>&1
mv -f .libs/source.lo source.lo
/bin/sh ../libtool --mode=link gcc  -g -O2  -o library.la source.lo -lm 
rm -fr .libs/library.la .libs/library.* .libs/library.*
ar cru .libs/library.al source.lo
ranlib .libs/library.al
creating library.la
(cd .libs && rm -f library.la && ln -s ../library.la library.la)
make[1]: Leaving directory /tmp/lib
\end{verbatim}

Next, the build enters the `lib' subdirectory to build `library.la'. The `configure' preprocessor macros are passed on the command line, since no `config.h' was created by AC\_{}CONFIG\_{}HEADER: 


Here, `main.c' is compiled (not to a Libtool object, since it is not compiled using libtool), and linked with the `liberror.la' Libtool library: 

\begin{verbatim}
gcc -DHAVE_CONFIG_H -I. -I.  -I./lib  -g -O2 -c main.c
/bin/sh ./libtool --mode=link gcc  -g -O2  -o convenience  main.o \
liberror.la 
gcc -g -O2 -o .libs/convenience main.o ./.libs/liberror.so -lm \
-Wl,--rpath -Wl,/usr/local/lib
creating convenience
make[1]: Leaving directory /tmp/convenience
\end{verbatim}

libtool calls gcc to link the convenience executable from `main.o' and the shared library component of `liberror.la'. libtool also links with `-lm', the propogated inter-library dependency of the `library.la' convenience library. Since `libreplace.la' and `library.la' were convenience libraries, their objects are already present in `liberror.la', so they are not listed again in the final link line -- the whole point of convenience archives. 


This just shows that it all works: 


\begin{lstlisting}[basicstyle=\scriptsize]
$ ls
Makefile      config.h       configure.in  install-sh   main.c
Makefile.am   config.h.in    convenience   lib          main.o
Makefile.in   config.log     error.c       liberror.la  missing
aclocal.m4    config.status  error.h       libtool      mkinstalldirs
config.cache  config.sub     error.lo      ltconfig     replace
config.guess  configure      error.o       ltmain.sh
$ libtool --mode=execute ldd convenience
        liberror.so.0 => /tmp/.libs/liberror.so.0 (0x40014000)
        libm.so.6 => /lib/libm.so.6 (0x4001c000)
        libc.so.6 => /lib/libc.so.6 (0x40039000)
        /lib/ld-linux.so.2 => /lib/ld-linux.so.2 (0x40000000)
$ ./convenience
cos (0) => 1
lt-convenience: ERROR: This program does nothing!
\end{lstlisting}

 Notice that you are running the uninstalled executable, which is in actual fact a wrapper script, See section 10.5 Executing Uninstalled Binaries. That is why you need to use libtool to run ldd on the real executable. The uninstalled executable called by the wrapper script is called lt-convenience, hence the output from basename().

Finally, you can see from the output of ldd, that convenience really isn't linked against either `library.la' and `libreplace.la'.
